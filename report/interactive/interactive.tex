\chapter{\is}
\label{chap:is}

\is er den sidste model for at belyse hvordan man kan implementere ti i \pycsp. Det har dog vidst sig at emnet er yderst begrænset dækket af litteraturen, og at de få definitioner af interactive tid vi har fundet ikke er en selvstændig tidsmodel men en definition af RTP med hard deadline, uden at være et kritisk system. \cite{?}. Vores udgangspunkt for denne model var istedet til brug i computerspil, men efter at have rådført os Med Kenny Erleben, underviser på  Det Danske Akademi for Digital Interaktiv Underholdning (DADIU), er vi kommet frem til der ikke findes en udbredt definition af tid i computerspil. Han fortæller desuden at  det ikke er i computerspilfirmaerne interesse at  oplyse om deres tidsmodel da de  betragte den som en  virksomhedshemmelighed. 

Vi vil derfor istedet på baggrund af en række praktiske eksempler indkredse hvad vi forventer et \is skal kunne bruges til og på  baggrund af eksemplerne finde en definition af \is. Som hvis det er muligt implementere. Dette kapitel adskiller sig dermed væsentligt fra de foregående to kapitler i og med tidsmodellen  ikke baserer sig på en fast definition givet af litteraturen og vi ikke kan udbygge på kendt viden.

\section{Eksempel}
Oprindeligt forstillede vi os at \is var defineret af computerspilindustrien som en metode til at planlægge hvordan spillet skal forløbe. Der findes dog ingen akademiske værker der fokusere på \is i spil, så vi vil i stedet opstille et hypotetisk eksempel. 

\subsection*{Computerspil}
Vi forestiller os et computerspillet der er bygget på \pycsp og hvor hvert element i spillet er en selvstændig proces. I dette computerspil ønsker man en fugl der flyver på tværs af skærmen. Fuglen skal starte på et givet tidspunkt og med en fast hastighed bevæge sig over skærmen. Fuglens flugt over skærmen udregnes af to typer processer, efter en model der minder om videokomprimering. Den første type, består af én højprioritets proces der står for udregne positionen af fuglen med sjældne mellemrum. Den anden type kan bestå af mange lavprioritetsprocesser. Hver lavprioritetsproces står for en egenskab, som  f.eks. at justere fuglen i forhold til dens position, optimere animationen af fuglen, tilknyttet fuglekvidre og andre ikke essentielle egenskaber.  Den højtprioriterede proces udføres sjældent men skal udføres, mens lavprioritetsprocesserne blot skal udføres hvis der er mulighed for det og ellers skal droppes.

Uden tab af generelitet vil vi i dette eksempel begrænse os til at fokusere på en høj og en lavprioritets proces. Højprioritetsprocessen står for at beregne positionen mens lavprioritetsprocessen står for at flytte fuglen i forhold denne position.




\subsection{Animeret ur}

Vi ønsker at animere et digitalur. Uret består af 4 cifre, og en gang i minuttet skal minuttallet tælles op. Denne begivnehed kan være længe undervejs, men det skal sikres at uret 

\subsubsection{WSN}
en radio skal være aktiv hele et givet tidsrum. f.eks fra t= 5 til t = 10. og hverken mere eller mindre.

\subsubsection*{Udskæring af gris}
Grisen ankommer til til t=5 og skal være færdig til t =10 , men det er frit hvornår inden for dette tidsrum det gøres.
Dårligt.

vise e ur. Hvert minut skal et tal skifte. 

  \section{Beskrivelse/teori}
  \section{Design og implementering}
  \section{Evaluering}
  \section{Fremtidigt arbejde}
  \section{Opsummering}
