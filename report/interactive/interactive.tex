\chapter{Interaktiv planlægning}
\label{chap:is}

Interaktiv planlægning (\is) er det sidste anvendelsesområde vi vil belyse med henblik på at indføre tid i \pycsp. Det har dog vist sig at emnet kun er sparsomt berørt i litteraturen, og der er ikke en klar definition på hvad det er og hvor det anvendes. 
%og at de få definitioner af \is vi har fundet, ikke er et selvstændigt anvendelsområde, men en definition af RTP med en hard deadline uden at være et kritisk system. \cite{?}. 
Vi forventede at dette anvendelsesområde var brugt i forbindelse med computerspil, men efter at have studeret litteraturen, og have rådført os med Lektor Kenny Erleben, der underviser på Det Danske Akademi for Digital Interaktiv Underholdning (DADIU), er vi kommet frem til der ikke findes en udbredt definition af \is. Han fortæller desuden at en af begrundelserne for at computerspilfirmaerne ikke er interesseret i at oplyse om deres metoder til at planlægge begivenheder i spil, er at det  betragtes som en  virksomhedshemmelighed. 

Vi vil derfor i stedet på baggrund af en række praktiske eksempler indkredse hvad vi forventer \is skal kunne bruges til, og på baggrund af eksemplerne se på muligheden for at implementere en model, så eksemplerne kan implementeres i \pycsp. Dette kapitel adskiller sig dermed væsentligt fra de foregående to kapitler, i og med tidsmodellen ikke baserer sig på en fast definition givet af litteraturen og vi ikke bygger på kendt viden.

\section{Eksempler}
Vi har valgt to scenarier som vi forventer med fordel kan benytte \is. Det første er en repræsentation af et ur, hvilket vi mener er det simpleste eksempel der kan gøre brug af \is. Det andet eksempel er en del af et computerspil, som er vores forventede primære anvendelsesområde for \is. 

\subsection{Et ur}
Det første eksempel på \is er en repræsentation af et digitalur. Uret består af fire cifre, og en gang i minuttet skal minuttallet tælles op. Det er et  krav at en opdatering af uret skal sætte uret til at vise det korrekte tidspunkt. Vi forestiller os at uret er en del af et større system hvor der er andre ressourcekrævende begivenheder der er vigtigere at få udført end opdateringen af uret. Opdatering af uret foregår ved at planlægge en begivenhed til hvert minut, der specificerer hvad uret skal vise. Da det har en lav prioritet, er der ikke er nogen garanti for at denne begivenhed indtræffer i det minut den er planlagt til. Derfor skal begivenheden  bortkastes såfremt den overskrider sin deadline. Deadlinen vil være det sidste sekund inden et nyt minut begynder, for at sikre den krævede korrekthed. 

Uret er repræsentativt for \is da der planlægges en række begivenheder der skal foregå i fremtiden, og hvor hver begivenhed også har tilknyttet en deadline.

\subsection{Computerspil}
Et andet eksempel vi mener repræsenterer  \is, er mere realistisk og bunder i vores oprindelige forventning om at \is  var defineret af computerspilindustrien, som en metode til at planlægge hvordan spil kan forløbe. Uden deres definition af \is vil vi i stedet opstille et hypotetisk eksempel. 

Vi forstiller os et computerspil, der er skrevet i \pycsp og hvor hvert element i spillet er en selvstændig proces. I dette computerspil skal der være en fugl, der jævnligt flyver på tværs af skærmen. Fuglen skal starte på et givet tidspunkt og med en fast hastighed bevæge sig over skærmen. Fuglens flugt over skærmen udregnes af to typer processer, baseret på en model der minder om videokomprimering. Den første procestype, er en højprioritets proces der står for udregne positionen af fuglen med  et fast interval. Den anden procestype kan bestå af mange lavprioritetsprocesser. Hver lavprioritetsproces står for en egenskab ved fuglen, som  f.eks. at justere fuglen i forhold til dens position, optimere animationen af fuglen, tilknytte fuglekvidren og andre ikke essentielle egenskaber. Den højtprioriterede proces udføres sjældent men er essentiel at få udført, mens lavprioritetsprocesserne blot skal udføres hvis der er mulighed for det og ellers skal droppes.

Uden tab af generelitet vil vi i dette eksempel begrænse os til at fokusere på en høj og en lavprioritets proces. Højprioritetsprocessen står for at beregne positionen mens lavprioritetsprocessen står for at flytte fuglen i forhold denne position.

\section{Beskrivelse}
Ud fra eksemplerne kan vi se på hvilke egenskaber de har, og hvilke krav de derved stiller for at kunne håndteres i et programmeringssprog. Først og fremmest ligger det inden for realtid ligesom RTP. Ligeledes skal der være mulighed for at tilknytte en deadline til en begivenhed. Dette vil i vores opstillede eksempler være hard deadlines, men vi kan ikke udelukke at der findes eksempler hvor andre typer deadlines vil være fordelagtige. Eksemplet med computerspillet viser at der er behov for at tilknytte en prioritet, der er uafhængig af deadline, til en begivenhed. Dette har vi diskuteret som en mulig udvidelse til RTP i \cref{sec:deadlineFuture}. Disse egenskaber minder alle om dem der er er givet for RTP. Ud over disse skal vi også kunne tilknytte et starttidspunkt til en begivenhed. Det lægger sig mere op af \des med det forbehold at vi i \is ikke garanterer at en begivenhed sker på et bestemt tidspunkt, men tidligst på det angivne tidspunkt.  

Vi kan dermed se \is som en blanding af RTP og \des, med tidsmodellen og deadlines fra RTP, og startidspunkter fra \des. 

%\is arbejder ligesom RTP i realtid. Desuden minder de også om hinanden da man i begge modeller arbejder med begivenheder, der har tilknyttet deadlines. For det tredje viser computerspilseksemplet at en udvikleren skal kunne tilknyttes prioriteter til begivenheder, der skal sættes uafhængigt af deres deadline i \is. Dette nævnte vi som en mulig udvidelse til RTP i \cref{sec:deadlineFuture}.

%\is og \des minder også om hinanden da man i begge modeller skal kunne planlægge en begivenhed til at foregå ud i fremtiden. Men hvor \des kan garantere at begivenheden sker på præcist det angivne tidspunkt kan vi i \is kun garantere at det sker efter et givent tidspunkt.

\section{Design og implementering} 
Vi argumenterer i kapitel \ref{chap:des} for at planlægning af begivenheder til et givet tidspunkt, kan tolkes som venten indtil tidspunktet. Vi kan derfor se \is som RTP med den yderligere mulighed at kunne vente. Som nævnt ovenfor vil det også være et krav at vi kan sætte en prioritet på en proces uafhængigt at processens deadline. 

Da kravene til \is rent teoretisk ligger meget tæt op ad de løsninger vi tidligere har beskrevet, vil vi også tage udgangspunkt i de allerede udviklede implementeringer. 

%I \des kom vi frem til at planlægningen af begivenheder ud i fremtiden også kan tolkes som en venten. Dermed kan man se på \is som en udvidelse til RTP, hvor man udover at have mulighed for at sætte en deadline skal have mulighed for at vente. For at kunne bruge RTP til \is vil det desuden være hensigtsmæssigt at  udvide RTP så det er muligt for udvikleren at tilknytte en prioritet til begivenheden.

\subsubsection{\code{Now} og \code{Wait }}
Som vi beskrev i \des introducerede vi de to globale funktioner \code{Now} og \code{Wait} der hhv. returnerer den aktuelle tid og lader en proces vente i et givent tidsrum. Vi har ændret den interne implementering af funktionerne, så de bruger realtid. Ved at genbruge de samme funktioner, sikres en ensartet implementering af tid på tværs af TimedPyCSP, og man kan i vores øjne med fordel tilføje dem til alle \pycsp versionerne, for på den måde at ensrette de forskellige implementeringer. Hvis \code{Now} og \code{Wait} blev inkluderet i \code{greenlets}-versionen kunne de fjernes helt i \is, da de to baserer sig på den samme tidsmodel.

For at benytte realtid i forhold til diskret tid bruger vi pythons \code{time} modul. Genimplementering består derfor kun i at ændre funktionen \code{Now} til at bede \code{time}-modulet om tiden i stedet for \sched en.

\subsubsection{Prioriteter}
I computerspilseksemplet, har processerne forskellige prioritet dikteret af udvikleren. Derfor skal vi finde en metode, til at udvide RTP således at det kan håndtere processer hvor udvikleren også kan tilknytte en prioritet. Denne udvikler-prioritet er ikke det samme som den prioritet der allerede findes i RTP eksemplet, da denne prioritet er beregnet af \sched en.

I RTP versionen er \sched en baseret på EDF, som kun tager hensyn til processernes deadline når der beregnes prioriteter. Vi skal dermed bruge en anden skemaplanlægningsalgoritme, for også at kunne håndtere at udvikleren skal kunne definere prioriteter.
\inline{mangler stadig en del om prioriteter}
% dikterede valget af \sched at priorite


    
\section{Evaluering}
\section{Fremtidigt arbejde}
Noget om mulighederne for at lave det i proces-versionen af \pycsp.
\section{Konklusion}
