\section{Evaluering}
  \inline{Evaluering af hvordan eksemplet løses efter den valgte 
  implementering benyttes. Inkluderer test+performance}
\subsection{Test af korrekthed}
  Vi har igennem designet og udviklingen af \code{simulerings}-versionen brugt en Test Driven Development (TDD). I TDD starter man med at skrive tests til en ny egenskab der skal udvikles. Designet er implementeret korrekt når testene kan gennemføres korrekt. Dette medfører at vi løbende i forbindelsen med udviklingen af \emph{simulerings}-versionen har skrevet tests. Desuden har vi integreret alle tests fra \emph{greenlets}-versionen ind i \emph{simulerings}-versionen, således at tests skrevet til \emph{greenlets}-versionen også tester vores version. Resultaterne af testene kan ses i tabel \fxerror{TODO! testene virker ikke:(}
  
\subsection{Eksempler}
for at evaluere fordele og ulemper af simuleringsversionen af \pycsp, har vi genimplementeret eksemplerne fra \cref{sec:des-examples}, men denne gang med  brug af vores udviklede kode. Dermed kan vi sammenholde de to versioner, og se på fordele og ulemper ved simuleringsversionen.
ers

 
\subsection{Hajer og fisk på Wator}


\subsection{Kunder i bank}
Vi har implementeret de to bankeksempler ved brug af den nyudviklede simuleringsversion til \pycsp. Vi vil her se på de dele af koden hvor simuleringsversionen har haft størst indvirkning. 

I generatorfunktionerne er den funktion hvor man tydeligst kan  se forskellen på de to versioner. Overordnet set kan man se at greenlets versionen bruger 20 linjer kode hvor simulationen kun bruger 6 linjer. Den store forskel på de to funktioner er muligheden for i simulationen  at undgå brugen af barriere. Dermed skal processen ikke være aktiv i hvert tidsskridt, men kun i de tidsskridt hvor der skal produceres en kunde. Når processen kun er aktiv i de korekte tidsskridt kan vi undgå en række hjælpevariabler (\code{t\_event, time} og \code{numberInserted}). \code{t\_event}. Desuden kan vi slippe for begrænsningen der kræver at en tiden skal være et heltal (\cref{greenlets_generator} linje 29). Linjerne 35 til 41 er krævet da barrieren kræver det samme antal processer igennem hele kørslen(Se \cref{sec:barrierer}). I alt kan man derfor skrive en mere koncis proces med brugen af simulation. I resten af koden opnår vi lignene kodekonsentration, men vi har også udviddet eksemplet så vi gemmer antallet af kunder der befinder sig i banken i en \code{Monitor}. Dette er ikke strengt nødvendigt, men viser brugen af \code{Monitor} til simulering. Dette var naturligt forventet at koden kunne forbedres, da eksemplet er et typisk \des problem, men det er tilfredstillende at vi opnår de forventede forbedringer i implementeringen af simulationen.
\begin{lstlisting}[firstnumber=21, label=greenlets_generator, caption=Generatorprocessen for Greenlets versionen]
@process
def Generator(i,number,meanTBA, meanWT,
              customerWRITER,barrierWRITER,barrierREADER):
  t_event = 0
  time = 0
  numberInserted = 0
  while numberInserted<number:
    if t_event<=time:
      customerWRITER(Customer(name = "Customer%d:%02d"%
                     (i,numberInserted),meanWT=meanWT))
      t_event = time + round(expovariate(1/meanTBA))
      numberInserted+=1
    barrierWRITER(0)
    barrierREADER()
    time+=1
  retire(customerWRITER)
  try:
    while True:
      barrierWRITER(0)
      barrierREADER()
      time +=1
  except ChannelPoisonException: 
    return
\end{lstlisting}
\begin{lstlisting}[firstnumber=20, label=simulation_generator, caption=Generatorprocessen for Simulationsversionen]
@process
def Generator(i,number,meanTBA, meanWT, customerWRITER):
  for numberInserted in range(number):
    customerWRITER(Customer(name = "Customer%d:%02d"%(i,numberInserted),
                            meanWT = meanWT))
    Wait(expovariate(1/meanTBA))
  retire(customerWRITER)
\end{lstlisting}

\begin{lstlisting}[firstnumber=11, label=simpy_generator, caption=Generator funktion for simpy]
def generate(self,number,meanTBA,resource):         
    for i in range(number):
        c = Customer(name = "Customer%02d"%(i,))
        activate(c,c.visit(b=resource))              
        t = expovariate(1.0/meanTBA)               
        yield hold,self,t
\end{lstlisting}

En sammenligning af generatorfunktionen mellem \simpy og simulationsversionen viser ikke den store forskel i hvorledes  man  implementere eneratorfunktionen i de to sprog. I \simpy aktivere man kunden direkte, mens man i simulationsversionen sender kunden over en kanal. Der findes større forskel på de to implementeringer i selve kunde delen. Dette skyldes at i simuleringsversionen findes en bankproce, og denne dermed er delt på tværs af alle kunder, mens man i \simpy har en funktion per kunde. Denne forskel medvirker til at bankprocessen skal have styr på hvornår samtlige kunder ønsker at forlade banken igen hvilket kræver lidt mere kode. I hjertet af bankprocessen findes en \code{alternation}. Her afventer processen hele tide på enten at modtage en ny kunde, eller på at en kunden ønsker at forlade banken.
\begin{lstlisting}[firstnumber=39,float=hbtp, label=fig:simulation_bank, caption= Uddrag af bank processen i simulation]
    while True:
      msg = customerREADER()
      print "%94.0f: %s enter bank"%(Now(),msg.name)
      heappush(customers,(Now()+msg.waittime,msg))
      mon.observe(len(customers))
      while len(customers)>0:
        print "%94.0f: B: timeout is:%f"%(Now(),customers[0][0]-Now())
        (g,msg) = Alternation([(customerREADER,None),
                               (Timeout(seconds=customers[0][0]- Now()),None)
                             ]).select()
        if g == customerREADER:
          heappush(customers,(Now()+msg.waittime,msg))
          print "%94.0f: %s enter bank"%(Now(),msg.name)
        else:
          ntime,ncust = heappop(customers)
          print "%94.0f: %s left bank"%(Now(),ncust.name) 
\end{lstlisting}
\begin{lstlisting}[firstnumber=20 ,float=hbtp, label=fig:simpy_customer, caption=funktionen \code{visit} i \simpy]
     def visit(self,b):                                
        arrive = now()
        print "%8.4f %s: Here I am     "%(now(),self.name)
        yield request,self,b                          
        wait = now()-arrive
        print "%8.4f %s: Waited %6.3f"%(now(),self.name,wait)
        tib = expovariate(1.0/timeInBank)            
        yield hold,self,tib                          
        yield release,self,b                         
        print "%8.4f %s: Finished      "%(now(),self.name)
\end{lstlisting}
\begin{lstlisting}[firstnumber=33 ,float=hbtp, label=fig:simulation_bank2, caption=Bankprocessen hvor banken er en begrænset ressource. ]
  @process
def Bank(customerREADER):
  try:
      while True:
        print "%94.0f: B: waits for customer"%Now()
        customer = customerREADER()
        print "%94.0f: B: adding a customer  %s to queue"%(Now(),customer)
        Wait(customer.waittime)
        print "%94.0f: B: customer  %s exits queue"%(Now(),customer)
  except ChannelRetireException:
      print "%94.0f: B: got retire"%(Now())

\end{lstlisting}
I det avancerede eksempel som vist i  \cref{fig:simulation_bank2} indeholder bankprocessen mindre kode end i det simple eksempel. Det skyldes at banken ikke længere skal holde styr på samtlige kunder, men blot skal håndtere en kunde ad gangen mens resten af kunderne befinder sig i køen. Kundefunktionen i \simpy og bankprocessen i det avancerede eksempel minder meget om hinanden og en fordel ved \csp versionen af simulering at man kun foretager en \code{Wait} for at indikere af den begrænsede ressource er optaget, hvor man i \simpy versionen skal foretage tre kald. Først et kald for at få adgang til den begrænsede ressource, så et kald for at holde ressourcen i et tidsperiode, samt et sidste for at slippe ressourcen.

  
