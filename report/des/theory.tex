\subsubsection{PDES}
Når vi arbejder med CSP, hvis styrke bl.a. er samtidighed, er det oplagt at afsøge hvilket arbejde, der er lavet med henblik på samtidighed og paralellitet i \des. Dette forskningsområde kaldes parallel discrete event simulation(\pdes). Formålet med \pdes er en parallelisering af simuleringen for at kunne udnytte flere cpu'er, eller flere maskiner. Dette introducerer dog flere problemstillinger relateret til håndtering af tid, da tiden er nødt til at være lokal i stedet for global, og dermed kan varierer på tværs af simuleringen. Dette giver problemer som f.eks. modtagelse af beskeder fra fotiden, der kan tvinge en enkelt maskine til at rulle sin del af simuleringe tilbage til et tidligere tidspunkt. 

\begin{table}[ht]
	\centering
	\begin{tabular}{lrr}
	\toprule
	\mc{Periode} & \mc{DES} & \mc{PDES}\\
	\midrule
1970 til 1980 &   296 &2\\
1980 til 1990 & 1.460 &95\\
1990 til 2000 & 6.190 &1.260\\
2000 til 2010 &13.100 &1.210\\
\bottomrule
	\end{tabular}
	\caption{Publisering af artikler if. google scholar ved søgning på hhv. ``discrete event simulation'' og ``parallel discrete event simulation''}
	\label{tab:des}
\end{table}
%\subsection*{Noter til afsnittet}
%\begin{itemize}
%\tightlist
%	\item stokastisk varians i relation til M/M/1
%	\item Hvad bruges DES til? styrker/svagheder?
%	\item Henvis til DE-simulation.ps
%  \item Eksempel på problemer som løses med des
%  \item model vs. simulering
%\end{itemize}

\pdes har ikke vundet stort indpas i den videnskabelige verden, som man kan se af \cref{tab:des}. En grund til dette er, at når tiden kan køre parallelt, øges omkostningerne ved at administrere den mere komplekse tidsrepræsentation, hvilket resulterer i lavere hastighed, end når tiden kan holdes synkront på tværs af processerne.
Grundet den forholdsvis dårlige ydelse og deraf følgende manglende interesse, har vi også valgt at lade \pdes ligge, og fokusere udelukkende på almindelig \des indenfor simulering i diskret tid. 

\subsubsection{Barrierer} \label{sec:barrierer}
%\inline{Hele afsnittet skal skrives om for at få det generelle samlet i starten, og vores specifikke efterfølgende. Evt. have mindre om vores implementation}

I \des er tiden den samme for alle processer, og de skal derfor have en fælles tid, der fremskrives samtidigt for alle processerne. En global viden som tid kræver derfor synkronisering af alle 
processerne, og til denne koordinering og synkronisering af flere 
processer er  den mest brugte metode at introducere en barriere. Barrierer blev først introduceret i MPI \cite{mpi-barrier}, hvor den bruges til at 
sikre, at alle tråde venter i barrieren før de  fortsætter. 

I \csp kan man lave sin egen barriere ved at udnytte, at begge 
kanalender skal være klar, før der kan kommunikeres, og at en proces, der 
indgår i en kommunikation, derfor vil vente indtil den anden ende er klar. Ved hjælp af kanaler kan man derfor lave en simpel barriere 
 ved brug af kommunikation over kanaler.  En implementering af den simple 
barriere som en selvstændig proces kan ses i \cref{barrier-imp}.

\begin{lstlisting}[float, label=barrier-imp,caption=En barriere i \pycsp]
@proces
def Barrier(nprocesses, signalIN, signalOUT):
	while True:
		for i in range (nprocesses):
			signalIN()
		for i in range (nprocesses):
			signalOUT(0)
\end{lstlisting}
%
%Denne implementering af en barriere kræver, i modsætning til de fleste andre 
%implementeringer af barrierer\cites{mpi-barrier, crew}, to kald. Det første 
%sender en variabel til barriereprocessen, mens,
%det andet kald modtager en dummyværdi fra barriereprocessen. Det kræver derved 
%to kanaler at implementere barrieren. På den ene kanal er barrieren den eneste 
%der læser værdierne; en besked sendt på denne kanal vil derfor altid modtages 
%af barrieren. På den anden kanal er barrieren den eneste der skriver, og en 
%modtaget besked må derfor komme fra barrieren.
%
%Vi kan overbevise os om korrektheden af barrieren, da alle processerne først 
%går ind i barrieren ved at sende en værdi til barrieren. Hvis barrieren ikke er 
%klar, sikrer \csp at processerne venter indtil barrieren er klar til at modtage 
%værdierne. Først når barrieren har modtaget en værdi fra alle processerne, 
%begynder barrieren at sende sin værdi, og det er først når en proces modtager 
%denne værdi fra barrieren at den må fortsætte. Når en proces modtager værdien 
%fra barrieren fortsætter den og man kan risikere at den ønsker at gå ind i 
%barrieren inden denne har sendt sin værdi til alle processer for at frigive dem.
%Processer der ønsker adgang til barrieren vil da gå i stå, idet de prøver at 
%sende til barrieren før den er klar til at modtage. Først 
%når barrieren har signaleret til alle processer at de må fortsætte, læser den 
%på kanalen for at accepterer processer der ønsker at tilgå barrieren. Det er 
%denne egenskab fra \csp der giver os garanti for at en proces der netop er 
%frigivet fra barrieren ikke går ind i den igen og derved risikerer at komme 
%foran. 
%
%En ulempe ved denne simple barriere er at antallet af processer skal være 
%konstant gennem hele kørslen.
%Vi vil senere se på et bankeksempel hvor dette problem opstår (\cref{bank-eksempel}). Her ville nogle af 
%processerne kunne slutte tidligt, men må fortsætte med  at kalde barrieren, indtil alle processerne er klar til at slutte.
%Man kunne ændrer barriereprocessen, så man dynamisk kan ændre på antallet af processer der 
%skal synkroniseres. I de fleste implementationer af barrier er dette en mulighed, men til vores simple illustration, har vi valgt ikke at implementere det.

Barrierer er en meget effektiv metode til at synkronisere processer, der kører 
parallelt, og er brugt flittigt i MPI. I \csp findes der dog en konflikt i brugen 
af barrierer, selvom det er nemt at implementere. Problemet er at hver proces bør fungere i isolation, og den eneste interaktion mellem processerne er via kanalerne. Derfor virker introduktionen af barrierer og kald til disse kunstig i \csp. 

\citeauthor{crew} beskriver brugen af barrierer som:
\begin{otherlanguage}{english}
\mycite[1]{crew}{[\ldots] where the barriers may be used to maintain global and/or localised models of time and to synchronise safe access to shared data[\ldots]}
\end{otherlanguage}
Barrierens berettigelse i CSP er derfor at  introducere tid og at kunne bruge delt data. I \csp bør der ikke være delt data mellem processerne, men derimod kun lokalt data. Hvis data er delt pga. den arkitektur \csp er implementeret på, bør dette abstraheres væk og udnyttes internt i kanalerne. At introducere hjælpemidler for at styre delt data, er derfor at tilskynde til en forkert brug af \csp. Tiden er den anden begrundelse for at benytte barrierer.
Barrierer giver dog kun en  primitiv model for tid, og vi vil vise, at med brugen af \des får man et stærkere værktøj, der blandt meget andet også kan erstatte brugen af barrierer.

