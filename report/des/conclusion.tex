\newpage\section{Opsummering}
Vi har i dette afsnit gennemgået \des og redegjort for dens anvendelsesområder, og kommet frem til at en \des grundlæggende skal have tre egenskaber: En repræsentation af diskret tid, der kan styres af simuleringen, en liste over fremtidige begivenheder og muligheden for at opsamle statistisk data. 

På baggrund af de tre grundlæggende egenskaber har vi defineret to eksempler der viser behovet for en bedre metode til at foretage \des i \pycsp. 
De to eksempler er begge implementeret med og uden brug af vores \des løsning. Vi har redegjort for at uden brug \des, er det nødvendigt at benytte barrierer til at repræsentere tid, og at denne metode bryder med det grundlæggende princip i CSP om ikke at have delte datastrukturer. 

På baggrund af eksemplerne og de grundlæggende egenskaber, har vi lavet en fuldstændig implementering af \des i \pycsp.
Vores implementering løser problemet med at tid kræver en delt datastruktur, og derudover illustreres det, at det er simpelt for en udvikler at benytte vores implementering. 
 Vi har vist  at en implementering af eksemplerne med brug af vores løsning kræver væsentligt mindre kode, end hvis de var implementeret direkte i \pycsp. Generelt resulterer vores løsning derfor i en implementering der er mere overskuelig og hvor koden primært udtrykker det konkrete problem der skal løses. 

De to eksempler er af grundlæggende forskellig karakter, og taget med for at vise hvordan vores løsning håndterer to meget forskellige simuleringer. Der er forskel på hvor meget de respektive eksempler drager nytte af vores implementering, men begge eksempler kan repræsenteres bedre med, end uden, vores implementering. 

