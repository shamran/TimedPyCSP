\chapter{Introduktion}
  \section{Problem}	 
  \section{Kontekst - Baggrund og motivation}
  \section{Fremgangsmåde}
  \section{``Summary of contributions''}
  \section{Termer}


sammenligning af simulation vs. modellering
  simulering er en implementaiton af modellen
  styrke er samspillet mellem flere elementer/modeller


simulering - dynamisk - tid (time/space)


Simuleringer har længe været et værdifuldt værktøj til at klarlægge hvordan et 
system fungerer og er specielt brugbart til at repræsentere systemer hvis 
tilstand ændres over tid, eller der er interaktion mellem flere systemer. En 
matematisk model af disse systemer vil ofte være væsentligt mere kompleks og 
kan være svær at overskue lige så let som en simulering af samme system. 

Simuleringer foretages ofte af andre videskabsfolk end dataloger, og det er 
derfor vigtigt at de kan repræsenteres i et sprog som er let tilgængeligt og 
minimerer sandsynligheden for at begå fejl i konstruktionen af simulationen.  
Til dette formål er programmeringssproget Python oplagt da det netop fokuserer 
på let tilgængelighed og høj produktivitet for udvikleren. 

Det er oplagt at benytte CSP\cite{hoare-csp} til at repræsentere en simulering.  
I CSP er interaktionen mellem forskellige processer/systemer eksplicit, og på 
grund af modulariteten er det nemt at konstruere komplekse systemer ud fra 
mindre enkle systemer. 



Der er derfor et stort behov for at kunne udtrykke simuleringer på en let 
tilgængelig måde (enter Python?) -> PyCSP




  

