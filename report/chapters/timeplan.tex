\chapter{Tidsplan}
Når et kapitel er færdig skal det læses igennem af os begge og gennemdiskuteres.
\begin{list}{}{}
\tightlist 
\item [8/2] \des færdig.
\item [20/2-28/2] Vinterferie.
\item [29/2] \ds færdig.
\item [3/5] \is færdig.
\item [10/5] Første fulde gennemlæsning færdig, med fokus på sammenhæng af kapitlerne
\item [14 dage buffer.]
\item[25/5] Undersøg med informationen om de har åbent d. 31/5.
\item [25/5-27/5] Sidste gennemlæsning færdig med fokus på små rettelser.
\item [27/5 -30/5] sidste opsætning, printning, og indbinding.
\item [31/5] Aflevering. 
\end{list}\
\begin{tabular}{m{0.5cm}m{4cm}}
\hline  
\multicolumn{2}{m{4.5cm}}{\textbf{Status på \des:}} \\
\hline
100\% & Kode.  \\ 
70\% & Eksempler.\\
0\% & Beskrivelse og Teori.\\
0\% & Design og  implementering. \\
0\% & Evaluering. \\
0\% & Fremtidigt arbejde. \\
0\% & Opsummering. \\ 
\hline
\end{tabular}
\quad
\begin{tabular}{m{0.5cm}m{4cm}}
\hline  
\multicolumn{2}{m{4.5cm}}{\textbf{Status på Deadline scheduling:}} \\
\hline
0\% & Kode.  \\ 
0\% & Eksempler.\\
0\% & Beskrivelse og Teori.\\
0\% & Design og  implementering. \\
0\% & Evaluering. \\
0\% & Fremtidigt arbejde. \\
0\% & Opsummering. \\ 
\hline
\end{tabular}\\
\begin{tabular}{m{0.5cm}m{4cm}}
\hline  
\multicolumn{2}{m{4.5cm}}{\textbf{Status på interaktive scheduling:}} \\
\hline
0\% & Kode.  \\ 
0\% & Eksempler.\\
0\% & Beskrivelse og Teori.\\
0\% & Design og  implementering. \\
0\% & Evaluering. \\
0\% & Fremtidigt arbejde. \\
0\% & Opsummering. \\ 
\hline
\end{tabular}
\subsection*{Ugentlige deadline}
\textbf{Deadline d. 18. jan.}
\begin{itemize}{}{}\tightlist
\item Skrevet barrierer afsnit færdigt, så det er generisk og kan bruges af begge eksempler.
\item Wator overordnet og Wator uden tid skal være færdig.
\item Bank overordnet og uden tid skal være færdigt.
\item Simon skal have begyndt på design og implementering.
\end{itemize}