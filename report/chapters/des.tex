\chapter{\des}
  \section{Beskrivelse/teori}
    \fxnote[inline]{Beskrivelse af tidsmodellen, teorien omkring den og 
    hvor/hvad den benyttes til. Teori: henvisning til litteratur, bl.a.  
    matematik/beviser for modellen}
    
\subsection*{Barrierer}\label{barrier}
For at modellere eksemplet med kunder i en bank i PyCSP, skal der foretages en 
række valg. Først og fremmest skal det besluttes hvad der skal være processer, 
og hvordan  kanalerne skal forbindes. I \des findes der i modsætningen til 
PDES \fixme{referencer} en global tid og alle processerne skal derfor have  en 
fælles tid der tæller op samtidigt. Den mest brugte metode i CSP sammenhæng er 
at bruge en barrier \vref{barrier-imp}.

\begin{lstlisting}[label=barrier-imp,caption=En barrier i PyCSP]
@process
def Barrier(nprocesses, signalIN, signalOUT):
	while True:
		for i in range (nprocesses):
			signalIN()
		for i in range (nprocesses):
			signalOUT(0)
\end{lstlisting}

Barrier blev introduceret i MPI, og sikrer at alle tråde har nået dette punkt 
før de må fortsætte. I PyCSP er det trivielt at have  denne egenskab, da 
processer til begge kanalender er klar, og med to kanaler kan man sikre at 
ingen proces modtager et signal før alle processer har sendt et signal til 
barrieren.
\fxnote[inline]{fortsættes med ulemper med barriere og generelt hvorfor vi 
skal holdes os fra det. noget med at man skal kalde barrieren to  gange}

\section{Eksempel}
\fxnote[inline]{Beskrivelse af eksemplet og hvordan det er relateret til problemet/modellen. Beskrivelse af løsning uden vores tid}


\subsection{Hajer og fisk på Wator} Som eksempel på en \des har vi valgt at tage 
udgangspunkt i det scenarie som A. K. Dewdney
beskrev i artiklen \cite{wator}. Artiklen beskriver den
fiktive planet Wator, der har form som en torus og er fuldstændig
dækket af vand. Verdenen er inddelt i felter som beskrevet på side
20 i \cite{wator}. Disse felter kan være tomme, indeholde en
fisk eller en haj. Følgende karakteristika beskriver fisk og hajers
opførsel.


\begin{itemize}
\item[Fisk]
Lever af plankton, en ressource som er uendelig. Hvis der er en ledigt 
tilstødende felt, bevæger en fisk sig til dette felt. Hvis der er flere ledige 
felter vælges et tilfældigt. Såfremt en fisk overlever 3 tidsskridt forplanter 
den sig.
\item[Hajer]
Såfremt der er fisk i et eller flere tilstødende felter, vil hajen bevæge sig 
til et af disse felter og spise fisken. Hvis der er ikke er nogen fisk i et af 
disse felter flytter hajen sig til et tilfældigt valgt ledigt felt. Hvis en haj 
ikke spiser i 3 tidsskridt dør den. Overlever den i 10 tidsskridt forplanter 
den sig.
\end{itemize}


For hvert tidsskridt vil alle fisk og hajer udføre en handling ud fra
ovenstående opførsel.

Til at initiere systemet skal der defineres en størrelse af verdenen,
samt hvor mange fisk og hajer der er til stede fra start. Disse fisk og
hajer placeres tilfældigt i verdenen.

Såfremt de initielle parametre understøtter en bæredygtig bestand
forventer vi at se bestanden af henholdsvis fisk og hajer oscillerer
afhængigt af hinanden.


\textbf{Uden tid}
\fixme[inline]{bedre overskrift}
For at simulere Wator verdenen i et CSP-system hvori tidsbegrebet ikke er 
introduceret, er vi nødt til at udføre en synkronisering af de enkelte 
processers arbejde. Denne synkronisering kan ske ved brug af barrierer, hvor 
alle processer udfører en handling og mødes i barrieren før de fortsætter.
Vi har valgt at basere vores model på \cite{crew}, hvor verdenen repræsenteres 
som en delt datastruktur og adgangen til denne styres med barrierer. I vores 
model er hver proces derved ansvarlig for en del af verdenen og tilgangen til 
den delte datastruktur sker ud fra CREW-princippet (Concurrent Read, Exclusive 
Write) som angivet i artiklen, og styres vha. af barrierer. 

Vi deler verdenen lodret, hvor hver proces styrer en verdensdel. Hver proces 
gennemgår sin verdensdel og udfører en mulig handling for hver fisk og haj, dog 
vil fisk og hajer i de sidste to kolonner i hver verdensdel ikke bliver 
opdateret på nuværende tidspunkt. Dette skyldes at henholdsvis processen selv 
og den umiddelbar titl højre for den, har skriverettigheder til disse to 
kolonner.
Når denne opdatering er fuldført mødes processerne i en barriere, hvorefter 
hver proces opdaterer de to sidste kolonner. Herefter mødes processerne igen i 
barrieren, og når alle kolonner er opdateret, kan der foretages en 
visualisering af de udførte opdateringer. Disse skridt udføres et forudbestemt 
antal iterationer. 



\subsection{Kunder i en bank} Et klassisk eksempel inden for \des er at simulere  
en række kunder der alle ankommer til en butik, hvor de skal serviceres. Dette 
simple problem kan bruges til at modellere mange forskellige 
problemstillinger, som hvordan flowet ændrer sig hvis man varier en parameter 
i systemet. Programmeringssproget SimPy har som et af deres eksempel, en 
simulering af kunder i en bank. SimPy bruger dette som en gennemgående 
eksempler hvor de løbende udvider deres model for at vise forskellige 
egenskaber ved deres model. For at nemt at kunne sammenligne SimPy  med vores 
model vil vi bruge dette eksempel.

I det simple tilfælde af eksemplet ankommer der en kunde til banken på på et 
tilfældigt tidspunkt. Hun opholder sig i banken i et tidsrum, hvorefter hun 
igen forlader banken.

I det simple eksempel kan der ikke uddrages meget information, da banken ikke 
består af en begrænset ressource som kunderne skal tilgå.  Eksemplet er derfor 
også udvidet med en service disk, hvor alle kunderne skal betjenes af en 
servicemedarbejder. Alle kunder ankommer til banken i tilfældig orden og 
stiller sig i kø til at blive serviceret. Dette  svare til en M/M/1\fixme{hvad  
er en m/m/1 kø} kø.
\textbf{Uden tid}\fxnote{bedre overskrift}
I SimPy er kunderne en process og det vil derfor være nærliggende ligeledes at 
modelere eksemplet i PyCSP med kunder som en proces. i Simpy kalder 
generatorfunktionen kunderne og kan kører den parallelt med sig selv. 
I PyCSP\fxnote{hvorfor?- kan man lave child processes der køre i samme 
parallel kald som dets parent?}, er standard metoden\fxnote{set fra 
kodeeksempler} at modelere mere statiske objekter, med en source og sink 
process og lade arbejdet flyttes mellem processerne. Derfor er modellen ændret 
så  generatorprocesen stadigt fungere som en source, men vi introducerer en 
bankproces som sink, og lader kunderne være arbejdet der flyttes mellem dem. 

Mens SimPy  kalder kundeprocessen og lader denne stå for håndteringen af 
kunden og tiden hun befinder sig i banken, har bankprocessen i PyCSP ikke 
mulighed for at kende tid og skal derfor selv holde en liste med kunderne 
i banken og  til hver tidskridt vide hvilke af kunderne der skal forlade 
banken. 

Tiden er igen modelleret ved brug af barrier se \autoref{barrier}, men 
i stedet for  at kalde barrierens to krævede kald, lige efter hinanden, som 
normalt for at være sikker på vi er i samme tidsskridt lader vi bank
processen gå ind i barrieren i starten af tidsskridtet, og så modtage kunderi, 
indtil banken modtager et kald om at forsætte til næste tidsskridt af 
barrieren.
\begin{lstlisting}[label=bank-alternation-imp,caption=modtag kunder eller 
	barrier i Bankproces]
while True:
		(g,msg) = Alternation([{
		barrierREADER:None,
    customerREADER:None
    }]).select()
		if g == barrierREADER:
			break
    elif g == customerREADER:
			heappush(customers,(time+msg.waittime,msg))
\end{lstlisting}
Dette er smart, og nødvændigt for at lade banken have mulighed for at modtage 
et vilkårligt antal kunder i samme tidsskridt, samt vide hvornår der ikke vil  
komme flere kunder. Vi kan resonere os til at barrieren stadigt virker efter 
hensigten da for at generatoren kan komme foran med et tidsskridt og sende en 
kunde i et forkert tidsskridt skal have kaldt begge kald til barrieren, men 
barriernen vil ikke modtage et kald fra nogle før det har kaldt bankprocessen, 
og derfor må generatoren vente i første kald til barrieren indtil banken har 
modtaget sit kald fra barrieren. 

  \section{Design og implementation}
    \fxnote[inline]{Beskrivelse af design med udgangspunkt i eksemplet}
    \subsection{Scheduler}
Med valget af greenletversionen som grundversionen, og med henblik på at hovedparten af vores ændringer vil være i scheduleren, vil vi kort gennemgå denne.

\begin{lstlisting}[firstnumber=132,stepnumber=5,numbers=left, float, label=fig:scheduling, caption=Uddrag af Scheduler.py i greenletsversionen.]
    def getInstance(cls, *args, **kargs):
        '''Static method to have a reference to **THE UNIQUE** instance'''
        if cls.__instance is None:
            # (Some exception may be thrown...)
            # Initialize **the unique** instance
            cls.__instance = object.__new__(cls)

            # Initialize members for scheduler
            cls.__instance.new = []
            cls.__instance.next = []
            cls.__instance.current = None
            cls.__instance.greenlet = greenlet.getcurrent()

            # Timer specific  value = (activation time, process)
            # On update we do a sort based on the activation time
            cls.__instance.timers = []

            # Io specific
            cls.__instance.cond = threading.Condition()
            cls.__instance.blocking = 0
\end{lstlisting}

 I \cref{fig:scheduling} ses et uddrag af initialiseringskoden. Dels findes der tre lister af processer som scheduleren har mulighed for at vælge imellem når der skiftes proces.  
 \begin{list}
 \tightlist 
 \item \code{new}: Initeres på linje 140, og består af processer som lige er blevet scheduleres for første gang.
 \item \code{next}: Initeres på linje 141, og indeholder de processer der er klar til at blive kørt, og som har været kørt før.  
 \item \code{timers}: Initeres på linje 147, og indeholder de processer der har tilknyttet en timeout. De skal først scheduleres på et senere tidspunkt og venter dermed blot. Hvert element i listen består både af processen samt et tidsstempel for hvornår processen skal genaktiveres. Denne liste bliver gensorteret hver gang der indsættes en ny proces.
 \item \code{blocking}: Initieres på linje 150, og er en variabel.Processer der venter på IO operationer, er ikke klar til at blive scheduleret, men heller ikke afsluttet. Scheduleren kan derfor ikke schedulere dem, men holdes styr på antallet af ventende processer vha. denne variabel, for at kunne afgører om Scheduleren skal afslutte eller afvente.
\end{list}

Når \sched en er startet, itererer den igennem alle tre lister, indtil de alle er tomme, og der ikke er nogle processer der er blokeret. Dette betyder at der ikke længere kan komme processer der ønskes at blive lagt på \sched en, og den kan dermed afslutte.

For at markere at vi ikke kun skal foretage en planlægning
af processerne, men foretage en simulering, har vi lavet en
\code{Simulation} klasse der arver fra \code{Scheduler}. Alle ændringer
vi skal foretage for at gå fra en almindelig \sched ~til en simulerings
\sched, vil således indkapsles i denne klasse, mens alt hvad de to
klasser har til fælles vil være isoleret i greenlets versionen af
\code{Scheduler} klassen. Dette har yderligere den fordel at man tydeligt kan se
at alle klasserne i simulation versionen arver en simulerings \sched ~og
ikke \code{Scheduler} fra greenletsversionen.

%\fxnote{dårlig overskrift}{\subsection{Repræsentation af tid}} Vi
%vil i dette afsnit gå i dybden med listen \code{timers} der findes i
%klassen \code{Scheduler}, samt se hvordan den kan inkorporeres i vores
%design.

\subsection{Tid i greenletsversionen} I \pycsp foregår kommunikation
kun når begge kanalender er klar dvs. når der både findes en
kanalende der vil skrive og en kanalende der vil læse. Hvis
kun en af kanalenderne er klar, vil den vente indtil der findes
minimum en kanalende af hver type, der er klar. Dette medfører
risikoen for at processen aldrig kommer videre, men går i deadlock.
I \pycsp har man derfor i \code{alternation} mulighed for at
tilknytte en timeout til en \code{guard}. Dette giver mulighed for
at en proces, kun er villig til at vente på kommunikation i en
given tidsperiode. 
\begin{lstlisting}[float=hbtp, label=Timeout,caption=Timeout i Alternation (fra dokumentationen til PyCSP)]
Alternation([{Timeout(seconds=0.5):None}, 
             {Cin:None}]).select()
\end{lstlisting}

I \cref{Timeout} ses et minimalt eksempel hvor processen kun er villig
til at læse fra kanalenden \code{Cin} i $0.5$ sekunder. Hvis ikke der
er modtaget en besked indenfor 0.5 sekundt, accepteres timeoutguarden
og processen er ikke længere villig til at læse fra \code{Cin}, og
fortsætter sin kørsel.

Tid er dermed blevet introduceret i \pycsp, men kun for at at have
mulighed for at tilknytte timeout til en \code{alternation}. Vi ønsker
at videreudvikle denne struktur til at håndtere tidsdelen for alle
processer, samt for at fungere med diskret tid, modsat den eksisterende
løsning hvor tiden er kontinuerligt.

\subsection{Timers}  
Vi forventer at brugen af
listen \code{timers}, vil øges betragteligt og at den gennemsnitlige
længde af listen vil stige, når der udvikles simuleringsproblemer. Dermed 
øges kravet om en effektiv implementation af \code{timers}\fxnote{skal 
vi evaluere om dette rent faktisk forbedrer ydelsen}. 

For at forbedre ydelsen af \code{timers} listen, 
ændrer vi den fra en almindelig liste, til en min"-hob. En hob har
flere fordele ved skemaplanlægning og er nævnt i introduktionen til Pythons implementation af en hob\fxnote{ref til http://docs.python.org/dev/3.0/library/heapq.html}. 

Da en implementation af en hob
allerede findes i Pythoni modulet \code{Heapq}, som er effektivt implementeret i C, vælger vi at bruge denne. Den eneste handling
der ikke er som standard er implementeret, er fjernelsen af et arbitrært element
fra hoben. Dette sker i den eksisterende løsning når en proces
aktivere et andet valg i \code{alternation} end timeout. I dette tilfælde skal
processen ikke vente på sin timeout, men elementet skal fjernes fra
\code{timers} listen. For at fjerne et element i en hob, må man som i
en normal liste lave en lineær søgning i hoben, og derefter genoprette
hob"-egenskaben i listen. Dette vil dog ikke tage længere tid end det
allerede tager da en fjernelse af en timeout i greenletversionen på nuværende
tidspunkt bruger en lineær søgning, til at finde elementet der skal
fjernes, og genoprettelsen af hobegenskaben også tager lineær tid \inline{ref til construction of heaps can be done in linear time using Tarjas algorithm.\\$http://en.wikipedia.org/wiki/Tarjan\%27s\_algorithm$}For at se en forskellen mellem greenletversion der bruger en liste og simulationsversionen der bruger en hob kan man sammenligne \cref{sched_timer} linje 205 med \cref{sim_timer} linje 126.


\subsection{Diskret tid} For at konverterer greenletversionen der knytter sig til reel tid, skal vi ændre de steder som bruger tid. Som vi har beskrevet tidligere er det eneste sted tid er introduceret i forbindelse med timeout og dermed i listen \code{timers}. Vi kan dermed nøjes med at ændre de steder i \sched en som indvolvere \code{timers}. Det første sted hvor timers indgår er i udvælgelsen af hvilken proces skal vælges(\cref{sched_timer}). Her sammenlignes på linje 204  første tidsværdi i timers, med det nuværende tidspunkt i time klassen. Hvis det nuværende tidspunkt er større end værdien i timers udvælges denne proces til at køre næste gang og fjernes fra timers listen.

Da tiden er diskret, og kræver et aktivt valg før den skifter kan vi tilføje en yderligere begændsning i forhold til greenletsversionen. For simuleringsversionen skal tiden være præcist det der er angivet i \code{timers}, før processen skal aktiveres, og ikke kun størrer end som angivet i greenletsversionen.  Simuleringsversionen af den del der foretager udvælgelser en proces fra  \code{timers} kan ses i \cref{sim_timer}. 

\begin{figure}[hbtp]
\begin{minipage}[c]{\linewidth}
\begin{lstlisting}[firstnumber=204, label=sched_timer, caption=Udvælgelse af proces fra listen timers (fra scheduling.py)]
if self.timers and self.timers[0][0] < time.time():
  _,self.current = self.timers.pop(0)
  self.current.greenlet.switch()
\end{lstlisting}
\end{minipage}
\begin{minipage}[c]{\linewidth}
\begin{lstlisting}[firstnumber=124, label=sim_timer, caption=Udvælgelse af proces fra listen timers (fra simulation.py)]
if self.timers and self.timers[0][0] <= Now():
  assert self.timers[0][0] == Now()
  _,self.current = heapq.heappop(self.timers)
  self.current.greenlet.switch()
\end{lstlisting}
\end{minipage}
\end{figure}



\subsection{Funktionerne Now og Wait}
 I Python kan man benytte modulet
\code{time}, hvis man ønsker at introducer begrebet tid. Med dette
modul kan man få af vide hvad klokken er. fra en brugers synsvinkel
repræsenteres tiden som kontinuerlig, og hver gang en bruger spørge
om klokken, fås et bestemt tidspunkt. Med computere findes tid som
kontinuerligt begreb ikke, men derimod er tiden internt repræsenteret
som diskrete tidsskridt. Størrelsen af disse tidsskridt varriere
af bla. hvilken computer programmet køres på og operativsystem.
Når vi ønsker at introducere \des skal det ikke ses som diskret
modsat kontinuerligt, men at de enkelte tidsskridt i \des er af
variable størrelse modsat i modulet \code {time} der har en konstant
størrelse. Da man i \code{time} modulet har en fast tidsskridt og
tid i det kontinuerlig tifælde også er inddelt i faste størrelse
som eks. sekunder, kan man med time modulet måle tidsintevaller der
korrespondere med den kontinuerlige tid. I \des findes der ikke en
sammenhæng mellem den kontinuerlige tid og dens egen repræsentation af
tid. For \des er tid derimod blot et tal der starter som 0, og stiger
i abitrærer tidskridt. Når tiden i \des på denne måde er afkoplet
en relation almindelig tid, kan man heller ikke snakke om at et tidsrum
har sekunder eller timer. I \pycsp kan man i timeout planlægge en
begivenhed til at ske om f.eks. 5. sekunder. I \des findes sekunder som
begreb ikke, men man kan angiver at når tiden er talt op med 5 skal
begivenheden ske. \inline{Skal dette splittes op og halvdelen skal i
teori?}

Når et problem modeleres i \des, vil der altid være behov for at
tilføje en sammenhæng mellem tid i problemet og et tid i modellen. Da
der der ikke findes en fast sammenhæng, skal modellen derfor eksplicit
definere om 5 sekunder skal repræsenteres som, at tiden internt i
\pycsp tælles op med 5, 0.5 eller 0.05.

Vi har valgt at repræsentere tiden som en intern variabel i \sched.
Dette kan vi gøre da \sched ~er en singelton og der findes derfor kun
en variabel med tid. For processer der ønsker at kende tiden har vi
introduceret funktionen \code{Now()} der returnere hvad tiden er på et
given tidspunkt.

\inline{På det teoretiske plan snakker vi om at planlægge
begivenheder, mens vi i implantationen snakker om Wait og at ''stalle''
en process }

I programmeringssproget \simpy lader man en proces vente ved at
foretage kaldet \code{yield}. Dette yield sørger for at processen ikke
fortsætter før et foruddefineret tidsrum er gået.

\begin{lstlisting}[firstnumber=11 , stepnumber=2, numbers=left,float=hbtp, label=yield, caption= Et yield i \simpy (Taget fra Bank05.py i eksemplet fra \simpy)] 
def visit(self,timeInBank): 
  print now(), self.name," Here I am" 
  yield hold,self,timeInBank print now(),
  self.name," I must leave" 
\end{lstlisting}

I \cref{yield} ses hvordan en kundeproces ankommer til banken,
printer tiden, foretager et yield, og når processen fortsætter
fra dette kald er tiden steget med værdien \code{timeInBank}.
Til slut printer processen igen tiden. Brugen af yield er knyttet
til implementeringen af \simpy og skyldes at \simpy implementere
hver process som en \code{corutine}. Vi skal i \pycsp have en
ligende mulighed for at lade en proces vente. Da dette allerede er
implementeret via timeout i greenlets versionen af \pycsp, kan vi
uden at ændre i den eksisterende \sched tilføje en ny funktion
\code{Wait} der fungere som timeout, men kan kaldes af processerne
på et vilkårligt tidspunkt. \inline{Dette skal måske ændres
da jeg ikke har talt med Rune om nødvendigheden af ''while now()<t)''. Og jeg har derfor ikke beskrevet i teksten hvordan funktionen virker.}

\begin{lstlisting}[firstnumber=20,float=hbtp, label=wait, caption=Wait i simuleringsversionen.] 
def Wait(seconds): 
  Simulation().timer_wait(Simulation().current, seconds) 
  t = Now()+seconds
  while Now()<t: 
    p = Simulation().getNext() 
\end{lstlisting}

Funktionen \code{Wait} er essentielt det eneste værktøj der skal til for at planlægge en begivenheder ud i fremtiden, og vi har på nuværenede tidspunkt en  simpel begivenhedssimulator der kører i reel tid. 

\subsection{Fra reel tid til diskret tid.}\label{sec:}
Vi ønsker sædvanligvis at en simulering, kan eksekveres uafhængigt af tiden der simuleres det vil sige at når der ikke sker flere begivenheder til et givent tidspunkt skal tiden fremskrives til det næste tidskridt hvor der sker en begivenhed og ikke kun med et fast tidsskridt. Modsat gælder det også at tiden ikke må tælles op før alle processer har indikeret at de ikke ønsker at foretage mere arbejde. 

I den eksisterende \sched ~ er tiden reel og indkrementeres derfor løbende uafhængigt af processernes tilstand. Dette kan illustreres med et eksempel; Proces 1 har startet en ny thread via et \code{io} kald, og er derfor blokeret. Proces 2 står i en \code{Alternation} med en timeout guard. Uafhængigt af tiden det tager proces 1 at komme ud fra sit blokerede kald, skal proces 2 vide at når timeout'en er indtrådt. Dette er implementeret i \cref{fig:blocking_sleep} på linjerne 242 til 251. For at nå disse linjer findes der processer der er blokeret samt processer der venter på en timeout. Nu startes en separat tråd der signalere \sched en, når næste begivenhed i \code{timers} listen indtræffer. \Sched en kan nu vente på et signal, som vil komme fra enten en blokeret proces eller den nyoprettede tråd.

Denne ekstra tråd til håndtering er tid i et blokeret kald er slet ikke nødvendigt i \des. For at tiden skal tælles op må ingen processer være blokeret; De skal i stedet enten have kaldt funktionen \code{wait} eller vente på kommunikation.  
Så længe der findes blokerede processer venter vi på dem, uden at tage hensyn til antallet processer i \code{timers}.

For at at simuleringen kan fortsætte skal tiden tælles op på et tidspunkt, og dette må gøres eksplicit at simulerings \sched en.  Kun i det tilfælde hvor der ikke findes nogle processer der kan planlægges vælger vi at tælle tiden op. Vi ved at der ikke kan foregå flere begivenheder til et tidsskridt når der kun findes processer i \code{timers} listen. Vi kan i det tilfælde finde tidspunktet for den næste begivenhed og sætte tiden til denne begivenhed. Følgende er implementeret i \cref{fig:sim_sleep}.
\begin{figure}[hbtp]
\begin{minipage}[c]{\linewidth}
\begin{lstlisting}[firstnumber=239, label=fig:blocking_sleep, caption=Uddrag af \sched en i \code{Scheduler}]
self.cond.acquire()
if not (self.next or self.new):
    # Waiting on blocking processes or all processes have finished!
    if self.timers:
        # Set timer to lowest activation time
        seconds = self.timers[0][0] - time.time()
        if seconds > 0:
            t = threading.Timer(seconds, self.timer_notify)
            # We don't worry about cancelling, since it makes no 
            #difference if timer_notify is called one more time.
            t.start()
            # Now go to sleep
            self.cond.wait()
    elif self.blocking > 0:
        # Now go to sleep
        self.cond.wait()
    else:
        # Execution finished!
        self.cond.release()
        return
self.cond.release()
\end{lstlisting}
\end{minipage}
\begin{minipage}[c]{\linewidth}
\begin{lstlisting}[firstnumber=158, label=fig:sim_sleep, caption= uddrag af \sched en i \code{Simulation}]
self.cond.acquire()
if not (self.next or self.new):
  # Waiting on blocking processes
  if self.blocking > 0:
    # Now go to sleep
    self.cond.wait()
  #If there exist only processes in timers we can increment
  elif  not (self.next or self.new or self.blocking): 
      if self.timers:
          # inc timer to lowest activation time
          self._t = self.timers[0][0]
      else:
          # Execution finished!
          self.cond.release()
          return
self.cond.release()  
\end{lstlisting}
\end{minipage}
\end{figure}

\subsection{Ting vi har stjålet fra \simpy.}
I vores implementering findes der i sagens natur i stort overlap med \simpy, som har været en inspirationskilde til hvordan et simuleringssprog kunne udvikles i Python. En del af arbejdet med \simpy har vi kunne bruge direkte i vore implementering, efter devisen om ikke at genskrive eksisterende god kode. Det drejer sig om funktionalitet til dataindsamling, bearbejdning og visualisering. I \simpy findes en \code{Monitor} klasse. Formålet med denne klasse er at gemme tid/værdi par. Dermed kan man efter endt simulering, analysere på hvordan værdierne ændret sig over tid.



  \section{Evaluering}
    \fxnote[inline]{Evaluering af hvordan eksemplet løses efter den valgte 
    implementation benyttes. Inkluderer test+performance}
  \section{Fremtidigt arbejde}
  \section{Opsummering}
