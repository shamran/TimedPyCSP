\chapter{\Des}
\inline{sammenligning af simulation vs. modellering.\\
  simulering er en implementation af modellen.\\
  styrke er samspillet mellem flere elementer/modeller.\\
  simulering - dynamisk - tid (time/space).} 

Simuleringer har længe været et værdifuldt værktøj til at klarlægge hvordan et 
system fungerer og er specielt brugbart til at repræsentere systemer hvis 
tilstand ændres over tid, eller såfremt der er interaktion mellem flere systemer. En 
matematisk model af disse systemer vil ofte være væsentligt mere kompleks og 
kan være svær at overskue lige så let som en simulering af samme system. 



Simuleringer foretages ofte af andre videnskabsfolk end dataloger, og det er 
derfor vigtigt at de kan repræsenteres i et sprog som er let tilgængeligt og 
minimerer sandsynligheden for at begå fejl i konstruktionen af simulationen.  
Til dette formål er programmeringssproget Python oplagt da det netop fokuserer 
på let tilgængelighed og høj produktivitet for udvikleren. 

Det er oplagt at benytte CSP\cite{hoare-csp} til at repræsentere en simulering.  
I CSP er interaktionen mellem forskellige processer/systemer eksplicit, og på 
grund af modulariteten er det nemt at konstruere komplekse systemer ud fra 
mindre, enkle systemer. 

\section{Beskrivelse/teori} \label{sec:des-teori}
\inline{Beskrivelse af tidsmodellen, teorien omkring den og hvor/hvad den 
    benyttes til. Teori: henvisning til litteratur, bl.a.  matematik/beviser 
    for modellen}
\inline{Noget med hvilke funktion der er krævet. (Kun en Wait og en Now)}


Indenfor simulering er \des en meget brugt metode til at modellere systemer. I 
\des anskues tid som diskrete tidsskridt som er uden kobling til realtid. I 
disse tidsskridt udføres en eller flere begivenheder, som ved afslutning kan 
føre det modellerede system over i en ny tilstand, og herved et nyt tidsskridt. 
Det er irrelevant hvor lang tid det tager at udføre en begivenhed målt i 
realtid, der sker først en overgang til en ny tilstand når alle begivenheder 
for et givet diskret tidsskridt er udført. Dette er årsagen til at der ikke er 
nogen kobling mellem den diskrete tid og realtid, og et diskret tidsskridt kan 
variere arbitrært i realtid. Begivenhederne der skal udføres af systemet kan 
enten være givet på forhånd, eller blive skemaplanlagt dynamisk under afviklingen 
af andre begivenheder. 
Afhængig af hvad der simuleres, kan man udtrække relevant information om systemet, f.eks. gennemsnitlig behandlingstid for et element, længden af køer i systemet, og den samlede aktivitetstid for hvert delelement i systemet. 


for at kunne konstruere en \des skal vi derved have følgende til rådighed; En repræsentation af tid til at styre hvornår vi skifter tidsskridt, en liste over begivenheder der skal udføres i hvert tidsskridt, samt mulighed for at opsamle statistisk data fra simulationen. 

    
\csp er lavet med henblik på processer der arbejder samtidigt og uafhængigt af hinanden. \Des er derimod ikke parallelt, men schedulere sine begivenheder lineært efter hvornår de sker. I sumuleringen løbes begivenhederne kronologisk igennem til der enten ikke er flere begivenheder eller simuleringen når et forud defineret tidspunkt.

\inline{Der findes et åbenlyst problem med introduktionen a \des i \csp, som vi kort berørte i cref sec:barrierer }
Alternativet til  \des  er \pdes\fxnote{ref} hvor processerne kører parallelt, og  tiden dermed kan løbe uafhængigt af hinanden. Dette introducere muligheden for større parallelitet, men samtidigt risikerer processerne ved kommunikation at modtage beskeder fra fortiden, som der skal tages hånd om, f.eks. ved at rulle tiden tilbage. Det har der vidst sig at omkostningerne ved at lade tiden kører parallelt er omkostningerne større end ved at holde tiden synkront på tværs af processerne.

\subsection*{Noter til afsnittet}
begivenheder - liste - statisk, dynamisk skemalægning \\
stokastisk varians\\
målepunkter - brugbare data - hvad kan vi få af information: kølængde, gennemsnitlig/min/max behandlingstid, \\
fordele\\
forklaring af hvad DES er\\
Beskriv modellen - tilstande, overgange, diskrete tidsskridt(events i sig selv tager ikke tid, tiden tikker mellem events)\\
Hvad bruges DES til? styrker/svagheder?\\
Henvis til DE-simulation.ps\\

\subsection{Barrierer} \label{sec:barrierer}
\fxnote*{RS: Jeg synes det er forkert at starte et afsnit med ``I modsætning 
til\ldots'' når det ikke er en reference til noget ovenstående}{I \des findes 
der i modsætningen til Parallel \des} \fixme{referencer} en 
global tid og alle processerne skal derfor have en fælles tid der tæller op 
samtidigt.  En global viden som tid kræver synkronisering af alle 
processerne\fxnote{ref}, og til denne koordinering og synkronisering af flere 
processer er den mest brugte metode at introducere en barriere.

Barrierer blev først introduceret i MPI\fxnote{ref}, hvor den bruges til at 
sikre at alle tråde venter i barrieren før de kan fortsætte. 

I \fxnote*{RS: Gælder vel generelt for CSP}{\pycsp} kan man udnytte at begge 
kanalender skal være klar, før der der kan kommunikeres og at en proces der er 
indgår i en kommunikation vil vente indtil den anden ende er klar før den 
fortsætter.  Ved hjælp af kanaler kan man derfor lave en simpel barriere 
trivielt ved brug af kommunikation over kanaler.  En implementering af en 
barriere som en selvstændig proces kan eksempelvis implementeres som i 
\cref{barrier-imp}.

\begin{lstlisting}[float, label=barrier-imp,caption=En barriere i \pycsp]
@proces
def Barrier(nprocesses, signalIN, signalOUT):
	while True:
		for i in range (nprocesses):
			signalIN()
		for i in range (nprocesses):
			signalOUT(0)
\end{lstlisting}

Denne implementering af en barriere kræver, i modsætning til de fleste andre 
implementeringer af barrierer\cites{mpi-barrier, crew}, to kald. Det første 
sender en variabel til barriereprocessen, mens,
det andet kald modtager en dummyværdi fra barriereprocessen. Det kræver derved 
to kanaler at implementere barrieren. På den ene kanal er barrieren den eneste 
der læser værdierne; en besked sendt på denne kanal vil derfor altid modtages 
af barrieren. På den anden kanal er barrieren den eneste der skriver, og en 
modtaget besked må derfor komme fra barrieren.

Vi kan overbevise os om korrektheden af barrieren, da alle processerne først 
går ind i barrieren ved at sende en værdi til barrieren. Hvis barrieren ikke er 
klar, sikrer \csp at processerne venter indtil barrieren er klar til at modtage 
værdierne. Først når barrieren har modtaget en værdi fra alle processerne, 
begynder barrieren at sende sin værdi, og det er først når en proces modtager 
denne værdi fra barrieren at den må fortsætte. Når en proces modtager værdien 
fra barrieren fortsætter den og man kan risikere at den ønsker at gå ind i 
barrieren inden denne har sendt sin værdi til alle processerfor at frigive dem.
Processer der ønsker adgang til barrieren vil da gå i stå, idet de prøver at 
sende til barrieren før den er klar til at modtage på den givne kanal. Først 
når barrieren har signaleret til alle processer at de må fortsætte, læser den 
på kanalen for at accepterer processer der ønsker at tilgå barrieren. Det er 
denne egenskab fra \csp der giver os garanti for at en proces der netop er 
frigivet fra barrieren ikke går ind i den igen og derved risikerer at komme 
foran. 

En ulempe ved denne simple barriere er at antallet af processer skal være 
konstant gennem hele kørslen.
Dette er f.eks. et problem i bankeksemplet (\cref{bank-eksempel}), hvor 
generatorfunktionen kan slutte lige efter at have genereret den sidste kunde.  
Her må den fortsætte med blot at kalde barrieren, indtil servicedisken har 
processeret alle kunderne, før hele programmet kan afslutte. Alternativt må 
barriereprocessen ændres, så den dynamisk kan ændre på hvor mange processer der 
skal synkroniseres. \inline{skal vi fortælle hvordan der kan gøres i \pycsp}
\fxnote{RS: Reference frem i teksten til eksempler der endnu ikke er introduceret}


Barrierer er en meget effektiv metode til at synkronisere processer der kører 
parallelt, og er brugt flittigt i MPI. I \csp er der dog en konflikt i brugen 
af barrierer da hver proces fungerer i isolation, og den eneste interaktion der 
skal være mellem processerne er når der kommunikeres via kanalerne. 
Introduktionen af barrierer og kald til disse virker derfor kunstig i \csp. 
\citeauthor{crew} beskriver brugen af barrierer som:

\mycite[1]{crew}{
\begin{otherlanguage}{english}
[\ldots] where the barriers may be used to maintain global and/or localised models of time and to synchronise safe access to shared data [\ldots]
\end{otherlanguage}
}

Barrierens berettigelse er derfor for at kunne introducere tid, samt for at kunne bruge delt data. I \csp bør der ikke være delt data mellem processerne, men derimod kun  lokalt data. Hvis der er data er delt pga. arkitekturen \csp er implementeret på, bør dette abstraheres væk men udnyttes internt i kanalerne. At introducere hjælpemidler for styre delt data, er derfor at tilskynde til en forkert brug af \csp. Tiden er den anden begrundelse for at benytte barrierer.\fxnote*{Skal hives et niveau op tekstmæssigt}{Men med brugen af barrierer til at modellerer tid, kan man får  en primitiv model for tid, og vi vil i vores speciale vise ved at inkluderer tid i stedet får man et stærkere værktøj der ud over en masse andet også kan erstatte brugen af barrierer.}


\subsection{Timeout} 
\inline{kausal tid.}
Med et simuleringsmiljø er det ikke nok
udelukkende kan kunne specifiere en timeout på kanalen ud i fremtiden,
det er også nødvendigt at man kan specificere at en proces er villig
til at kommunikere på et specifikt tidspunkt, samt at den kanalen
villig til at kommunikere på nuværende tidpunkt, men ikke ud i
fremtiden.

At ville kommunikere fra et given tidspunkt i fremtiden, svare til at   
vente uden at lave noget indtil det givne tidspunkt for så at forsøge 
at kommunikere fra tidpunktet indtil kommunktationen lykedes. Dette     
kan dermed laves helt uden en alternation, men ved blot med de to       
funktioner Wait(time), Cin().                                           

At ville være villig til at kommuikere til et given tidspunkt,         
medfører en ny problemstilling, da det kan introducere muligheden      
for deadlocks og nondeterminisme, \fxnote{Er det korrekt?}, for det     
er ikke defineret hvad tiden er hvis det ikke lykkedes at kommunikere   
i det givne tidsskridt. Der findes to muligheder, enten skal \sched     
en signalere at der ikke findes mere arbejde til dette tidsskridt, og   
lade processerne fortsætte i samme tidsskridt. Alternativt skal tiden  
tælles op til næste event, hvorefter processerne signaleres med at    
kommunkationen ikke lykkedes.                                           

\subsubsection{Timeout i samme tidsskridt} Hvis man vælger at
processerne skal signaleres i samme tidsskridt, vil en timeout
efterligne en SKIP guard, men hvor en alternation med en SKIP guard med
det samme kan fortsætte vil en timeout kræve at der blev ventet indtil
der ikke var flere processer der kunne processeres i samme tidsskridt.
Når der ikke er flere processer at vælge imellem skal de ventende
processer fortsætte. Her er muligheden enten at signalere hver proces
en af gangen, og lade den fortsætte inden næste proces signaleres,
eller alternativt, samtidigt at signalerer alle de ventende processerne.

Ved at signalere samtlige processer indføres muligheden for en
livelock. Dette sker hvis to eller flere processer er synkroniseret i
den samme tidsrytme. Et eksempel er hvis de alle i samme periode først
ønsker at modtage data, for derefter at sende data. Her vil alle
processerne uden timeouts vente på hinanden i en deadlock, men med en
timeout vil de alle fortsætte samtidigt og indgå i en ny deadlock,
PyCSP vil derfor med timeouts gå fra en deadlock til en livelock. En
livelock er defineret som hvor tilstandenen i ændres men der foregår
ikke reelt arbejde\inline{skal vi introducere begrebet deadlock,
livelock mv.}.

Ved at aktivere en proces ad gangen har denne proces mulighed for i
samme tidsskridt at indgå i kommunikation med en af de andre ventende
processer. Deadlocken kan man dermed løse og man har mulighed for at
undgå livelock problemet, ved samtidig signalering. Ulemperne er at der
skal foretages et valg om hvilke processer der signaleres med timeout,
og dermed har man risikoen for starvation\fxnote{ref?} hvis det er
den/de samme processer der bliver signaleret.

\subsubsection{Timeout i et efterfølgende tidsskridt} Ved at vælge at
en timeout først forekommer i et efterfølgende tidsskridt, har alle
processer haft den maksimale mulighed for at indgå i en kommunikation.
Desuden kan alle processerne signaleres samtidigt, da et alle timeouts
er overskredet. Man skal dog vælge enten at fortolke en timeout i samme
tidskridt som en timeout efter $\epsilon$, hvor $\epsilon$ sættes til
et vilkårligt lille tidsinterval. Da tiden springer i vilkårligt små
tidsskridt i en simuleringen, vil en fast størrelse af $\epsilon$
risikere at påvirke rækkefølgen af events. Alternativt kan man vælge
at lade tiden springe til nærste event, og der signalere alle timeouts.
Begge muligheder har dog implicit det problem at timeout var sat til
0 og ikke til hverken et vilkårligt lille $\epsilon$ eller anden
tidsenhed og man derfor ændrer på fortolkningen af kode.

\subsection{Vores valg}

Timeouts i Timed PyCSP er en af de vigtigste funktioner, da det disse
giver en der ikke er data, mtimeout processen fortsætte tildet under
antagelsen af problemer.

når tiden er tidspunkt til for et findes der allerede \inline{vi skal
kunne angive at vi ønsker at kommunikere men KUN i det tidsskridt vi er
i, og at hvis dette ikke kan lade sig gøre skal vi breake noget alla at
have en timeout på 0}



\section{Eksempel}
\inline{Beskrivelse af eksemplet og hvordan det er relateret til problemet/modellen. Beskrivelse af løsning uden vores tid}


\subsection{Hajer og fisk på Wator} Som et eksempel på en \des har vi valgt at 
tage  udgangspunkt i det scenarie som \citeauthor{wator}
beskrev i artiklen \citetitle{wator}\cite{wator}. \fxnote{google:lotka,  voltarre} Artiklen beskriver den
fiktive planet Wator, der har form som en torus og er fuldstændig
dækket af vand. Verdenen er inddelt i felter, som kan være tomme, indeholde en
fisk eller en haj\cite[20]{wator}. Følgende karakteristika beskriver fisk og hajers
opførsel.

\begin{itemize}
\item[\textbf{Fisk}]
Lever af plankton, en ressource som er uendelig. Hvis der er ét ledigt 
tilstødende felt, bevæger den sig til dette felt. Hvis der er flere ledige 
felter vælges et tilfældigt. Såfremt en fisk overlever 3 tidsskridt forplanter 
den sig.
\item[\textbf{Hajer}]
Såfremt der er fisk i et eller flere tilstødende felter, vil hajen bevæge sig 
til et af disse felter og spise fisken. Hvis der er ikke er nogen fisk i et af 
disse felter flytter hajen sig til et tilfældigt valgt ledigt felt. Hvis en haj 
ikke spiser i 3 tidsskridt dør den. Overlever den i 10 tidsskridt forplanter 
den sig.
\end{itemize}

For hvert tidsskridt vil alle fisk og hajer udføre en handling ud fra
ovenstående opførsel.
Til at initiere systemet skal der defineres en størrelse af verdenen,
samt hvor mange fisk og hajer der er til stede fra start. Disse fisk og
hajer placeres tilfældigt i verdenen.
Såfremt de initielle parametre for antal fisk og hajer understøtter en 
bæredygtig bestand forventer vi at se bestanden af henholdsvis fisk og hajer 
oscillerer afhængigt af hinanden.

Vi har valgt dette eksempel, da det er enkelt og let forståeligt, men samtidig 
introducerer problemstillinger omkring synkronisering når det paralleliseres.  
Disse problemstillinger optræder fordi en opdatering af hvert felt er afhængigt 
af de omkringliggende felter, og vil derfor være afhængig af felter fra andre 
processer i grænsetilfælde. Ud over at være afhængig af information fra andre 
processer, kan en opdatering også påvirke data hos andre processer.   

\subsubsection{Før introduktion af et tidsbegreb i \csp} For at simulere Wator 
verdenen i et \csp-system hvori tidsbegrebet ikke er introduceret, er vi nødt 
til at udføre en synkronisering af de enkelte processers arbejde. Denne 
synkronisering kan ske ved brug af barrierer, hvor alle processer udfører en 
handling og mødes i barrieren før de fortsætter.

Vi har valgt at basere vores model på \citetitle{crew}\cite{crew}\fxnote{Brian: 
Hvad er din holdning til litteraturhenvisninger.}, hvor verdenen repræsenteres 
som en delt datastruktur og adgangen til denne styres med barrierer. I vores 
model er hver proces derved ansvarlig for en del af verdenen og tilgangen til 
den delte datastruktur sker ud fra CREW-princippet (Concurrent Read, Exclusive 
Write)\cite[5]{crew}, der styres vha. barrierer.  

I en mere ren \csp-model ville man foretage en direkte udveksling af data mellem 
processerne, og undgå den delte datastruktur vi har i vores model.  Vi har 
valgt denne model frem for den mere rene \csp model, da den klarlægger brugen af 
barrierer bedre.  Det bliver meget eksplicit i koden hvornår hvilke dele 
opdateres, og hvornår processerne venter i en barriere.

Vi deler verdenen lodret, hvor hver proces styrer en verdensdel. Hver proces 
gennemgår sin verdensdel og udfører en mulig handling for hver fisk og haj, dog 
vil fisk og hajer i de sidste to kolonner i hver verdensdel ikke bliver 
opdateret på nuværende tidspunkt. Dette skyldes at der kan opstå en en race 
condition hvis disse to kolonner opdateres samtidig med de andre. Årsagen til 
dette er at både processen der er ansvarlig for kolonnerne, og processen 
umiddelbart til højre for den skal have mulighed for at flytte fisk og hajer 
ind i dette område. Hvis der i samme tidsskridt er mulighed for at flytte dem 
ud af området, vil opførslen afhænge af hvad der sker først.  Når denne 
opdatering er fuldført mødes processerne i en barriere, hvorefter hver proces 
opdaterer de to sidste kolonner. Herefter mødes processerne igen i barrieren, 
og når alle kolonner er opdateret kan der foretages en visualisering af de 
udførte opdateringer. Til slut mødes alle i en barriere igen før processerne 
kan begynde en ny iteration.

En repræsentation af opdelingen ses på \cref{fig:wator}. Her er verdenen 
opdelt mellem to processer, og de to kolonner mellem hver del opdateres i et 
separat tidsskridt.  

\begin{figure}
 \begin{center}
  \includegraphics[scale=0.75]{images/wator}
  \caption{Opdeling af verdenen mellem to processer. Der er for hver verdensdel 
  to kolonner som opdateres i et separat tidsskridt.}
  \label{fig:wator}
  \end{center}
\end{figure}


\begin{figure}[hbtp]
\begin{minipage}{\linewidth}
\begin{lstlisting}[label=code:wator-worldpart,caption=Uddrag af processen 
  \code{worldpart} i Wator]
  @proces
  def worldpart (part_id, barR, barW):
  
  ...
  
    while True:
      #Calc your world part:
      main_iteration()
      barW(1)
      barR()
      #Update the two shadowcolumns
      for i in range(world_height):
        for j in range(2):
          element_iteration(Point(right_shadow_col+j,i))
      barW(1)
      barR()
      #visualize have single access
      barW(1)
      barR()
\end{lstlisting}

\begin{lstlisting}[label=code:wator-visualize,caption=Processen 
  \emph{visualize} i Wator]
@proces
 def visualize(barR,barW):
   for i in xrange(iterations):
     barW(1)
     barR()
     barW(1)
     barR()
     pygame.display.flip()
     barW(1)
     barR()
   poison(barW,barR)
\end{lstlisting}

\end{minipage}
\caption[test]{\code{barW(1)} og \code{barR()} er henholdsvis skrivning og læsning til og 
fra en barriere som defineret i \cref{sec:barrierer}.}
\end{figure}
Afviklingen af programmet sker ved at et antal worldpart-, en 
visualize- og en barrier-proces køres parallelt. Af 
\autoref{code:wator-worldpart} og \pageref{code:wator-visualize}\fixme{ændrer pageref til vref i final}, ses det 
overordnede design af henholdsvis worldpart- og 
vizualize-processen.  Værd at bemærke er det store antal barrierekald i 
visualizeprocessen. Dette er til for at man kan benytte den samme barriere som 
wordpartprocessen. Alternativt kunne man have to barriereprocesser; en til 
synkronisering af worldpart med vizualize, samt en som 
worldpart-processerne brugte til synkroniseringen af opdatering af 
verdensdelen og de to sidste kolonner.\inline{hvad skal emphs og hvad skal med stå med code} 

\inline{konklusion hvordan var det at implementere og hvor stor parallelitet 
kan man opnå. }


\subsection{Kunder i en bank}\label{bank-eksempel}
Et klassisk eksempel inden for \des er at simulere  en række kunder der alle 
ankommer til en butik, hvor de skal serviceres. Dette simple problem kan 
udvides til at modellere mange forskellige problemstillinger, der berører 
hvordan flowet ændrer sig hvis man varier en eller flere parametre
i systemet. Programmeringssproget \simpy\fxnote{ref}har, som et af deres 
eksempler, en simulering af kunder i en bank. \simpy bruger dette som et 
gennemgående eksempel, hvor de løbende udvider modellen, for at vise 
forskellige egenskaber ved \simpy. For at kunne sammenligne \simpy  med vores 
implementering af logisk tid, vil vi implementere to af eksemplerne med kunder 
i en bank i \pycsp.

I det simple tilfælde af eksemplet ankommer kunderne til banken på 
tilfældige tidspunkter.  De opholder sig i banken i et tilfældigt 
tidsrum, hvorefter de igen forlader banken. I dette eksempel kan der ikke 
uddrages meget information, men det viser hvordan en simpel model er opbygget i 
hhv.  \simpy og \pycsp for at håndtere tid.

I det andet eksempel er modellen udvidet med en servicedisk. Her skal alle 
kunderne betjenes af en servicemedarbejder, som kun kan ekspederes en kunde ad 
gangen. Alle kunder ankommer igen til banken i tilfældig orden og stiller sig i 
kø for at blive serviceret. Dette  svare til en M/M/1\fxerror{hvad  er en m/m/1 
kø} kø. Det er igen et tilfældigt tidsrum som kunden bruger på at blive 
serviceret.  Dette er stadigt en meget simpel model, men med introduktionen af 
en begrænset ressource kan man uddrage information om den tilhørende kø, f.eks. 
kan der måles hvor lang det tager for hver kunde at blive betjent af 
servicemedarbejdere, samt hvordan køen opfører sig over tid. 

\subsubsection{Før introduktion af et tidsbegreb i \pycsp}
I \simpy er kunderne en proces og det vil derfor være nærliggende ligeledes at 
modellere eksemplet i \pycsp med kunder som en proces. Dette sker i  \simpy når
generatorfunktionen laver en kunde og lader denne køre parallelt med sig selv.  
\inline{hvorfor?- man kan lave child proces med spawn, der der køre i samme parallel 
kald som dets parent? - Hvorfor er dette ikke god CSP stil at bruge} er standard metoden\fxnote{set fra kodeeksempler}I \pycsp 
modellere mere statiske processer, gerne med en generator- og 
arbejderproces, og lade arbejdet flyttes 
mellem processerne igennem kanaler. I \pycsp modellen fungerer 
generatorprocessen som i \simpy, men vi introducerer en bankproces 
som arbejder, og lader kunderne være arbejdet der flyttes mellem dem. 

Mens \simpy kalder kundeprocessen fra generatoren og lader denne stå for håndteringen af kunden 
og den tid hun befinder sig i banken, kender bankprocessen i \pycsp ikke tid som 
sådan. Bankprocessen skal derfor selv vedligeholde en liste med kunderne, der findes inden i banken og til hver 
tidskridt vide hvilke kunder der skal forlade den. 

Tiden er igen modelleret ved brug af barrierer, se afsnit \cref{sec:barrierer}. I 
stedet for at have de to kald til barrieren som den kræver, lige efter hinanden, lader 
vi bankprocessen gå ind i barrieren i starten af tidsskridtet, og så modtage 
kunder, indtil bankeprocessen modtager det andet kald fra barrieren (se 
\cref{bank-alternation-imp}). Dette er nødvendigt for at lade banken have 
mulighed for at modtage et vilkårligt antal kunder i samme tidsskridt, samt 
vide hvornår der ikke vil komme flere kunder.  Vi kan ræsonnere os frem til at 
barrieren stadig virker efter hensigten ved  simpel indsigt.
For at generatorprocessen kan komme foran med et tidsskridt og sende en kunde i et forkert tidsskridt,
skal den have fuldført begge begge kald til barrieren, mens bankprocessen ikke 
har modtaget et kald fra barrieren. Barrieren vil dog ikke modtage kald fra 
nogle, før den har  kaldt bankprocessen. Derfor må generatorprocessen vente i 
sit første kald til barrieren, indtil banken har modtaget sit kald fra 
barrieren, før den kan sende en ny kunde.
Når bankprocessen har modtaget et kald fra barrieren er den ikke længere villig 
til at modtage kunder, før i det efterfølgende tidsskridt. Vi kan bruge samme 
analogi til at ræsonnere os frem til at bankprocessen ikke kan komme et 
tidsskridt foran generatorfunktionen, og derfor virker barrieren stadigt som 
forventet. 

\begin{lstlisting}[float=hbtp,label=bank-alternation-imp,caption=Modtage en 
  kunde eller barrierekald i Bankprocessen]
while True:
		(g,msg) = Alternation([{
		barrierREADER:None,
    customerREADER:None
    }]).select()
		if g == barrierREADER:
			break
    elif g == customerREADER:
			heappush(customers,(time+msg.waittime,msg))
\end{lstlisting}


I det mere avancerede eksempel hvor kunderne skal tilgå den samme begrænsede 
ressource dannes en kø. Denne kan i \pycsp modeleres på flere måder, afhængigt 
at hvilken proces der skal have ansvaret for at vedligeholde køen. En metode er 
internt i en proces at have en liste af kunder der venter, og lade det være 
processens ansvar at håndtere denne liste som en kø. Processen med ansvaret for køen kan 
så enten være den begrænsede ressource, eller en separat proces hvis eneste 
formål er at vedligeholde køen. For nyligt\footnote{d. 22. december 2009} er 
der i \pycsp blevet introduceret ''buffered channels''\cite{pycsp-r147}, og 
disse kan også bruges som en kø. Dermed kan man modellere sin ressource uden 
hensyntagen til håndtering af køen internt eller vha. andre processer, og blot 
lade processen læse fra kanalen når den er klar. Vi har i dette eksempel valgt 
at lade køen være repræsenteret ved en ''buffered channel'', da denne  kræver 
færrest linjer kode, men den kunne lige så godt være repræsenteret som en liste 
i servicedisk processen.


\subsubsection{Konklusion - evaluering - ?}
Ved at se på  implementeringen af de to eksempler af kunder i en bank sammenholdt med implementeringen i \simpy kan man se at de egner det sig godt til 
simulering, og fra eksemplet kan man se at der findes meget kode til 
vedligeholdes af de interne tidsvariabler, som \simpy ikke har behov for. Det drejer sig 
om kode der sørger for at hver proces kender tiden, samt at den er 
synkroniseret på tværs af processerne. Vi forventer derfor at koden skal kunne 
simplificeres i \pycsp med tid, så det bliver lige så simpelt at implementere som i 
\simpy. 

I \simpy findes begrebet ressource direkte i sproget, som en type og 
servicedisken er blot en instans af denne type, og ressourcen står selv for at 
håndtere køen. I \pycsp modelleres servicedisken som en separat proces, og 
bankprocessen er derfor reduceret til blot at sende kunden videre til 
servicedisken og lade kanalen håndtere køen. En ulempe ved brugen af ''buffered 
channels'' som en kø er, at kanalen har en fast størrelse på sin buffer, som 
angives når kanalen oprettes. Man kan dermed risikere en deadlock ved brug af ''buffered 
channels'' sammen med barrierer. Dette opstår hvis ikke processen i samme 
tidsskridt kan foretage transmissionen og kalde barrieren.
Dette kan løses med en \code{alternation} og \code{skip guard}, men så 
skal bankprocessen håndtere fejlede forsendelser og må nødvendigvis introducere en 
sekundær kø hvilket gør at introduktionen af en ''\code{buffered channel}'' er 
irrelevant. For undgå dette har vi i vores tilfælde valgt blot at 
angive en maksimal størrelse på bufferen som er større end det totale antal 
kunder banken modtager.\inline{husk at m/m/1 køer har uendelig længde, se $http://en.wikipedia.org/wiki/M/M/1\_model$ - kan bruges som argument for at have channels af uelndelig længde}

\subsection{Køer i CSP}
\inline{køer med buffers manuelle køer, og monitors sammen med køer.}




\section{Design og implementering}
%\inline{Beskrivelse af design med udgangspunkt i eksemplet}
For at designe en implementering af simulering i diskret tid i \pycsp, skal vi foretage en række ændringer i forhold til den nuværende implementering. Specifikt skal vi ændre på planlægningen og eksekveringen af processer, hvortil vi har brug for at kunne repræsentere en diskret tidsmodel. Vi vil i dette afsnit gennemgå de relevante problemstillinger og løsningsmodeller samt give et overblik over, hvordan vi har valgt at implementere ændringerne rent praktisk i koden. 


\subsection{Kodestruktur}  
Efter i kapitel \ref{chap:csp} at have valgt at udvide \emph{greenlets}"-versionen, skal vi vælge hvordan vi ønsker at videreudvikle koden. Vi forventer at genbruge store dele af koden fra \emph{greenlets}"-versionen, og kun foretage udvidelser på enkelte afgrænsede områder. Desuden ønsker vi at isolere vores ændringer fra den originale \emph{greenlets}"-versionen. Med denne isolation forventer vi, at hvis/når der sker tilrettelser af \emph{greenlets}"-versionen af \pycsp, vil man ikke skulle foretage de samme tilrettelser i vores version. 
Isolationen mellem de to versioner skal opnås via nedarvning, således at det fra en brugers synsvinkel ser ud til, at vores version er fuldstændigt sepereret fra \emph{greenlets}"-versionen.

Hver af de tre versioner har sin egen mappe i \pycsp og i hver af disse findes en tilhørende \code{\_\_init\_\_.py} fil, der fungerer som et manifest for den givne version. Vi opretter vores egen version kaldet \emph{simulation} og opretter også en tilhørende mappe på samme niveau som de andre versioner og med sin egen manifestfil. Manifestfilen kan nu bruges til at udvælge de funktioner, der skal tages direkte fra \emph{greenlets}-versionen, og hvilke funktioner, der skal udvides og som derfor vil ligge i den nye mappe.

\begin{lstlisting}[float=hbtp,label=fig:init,caption=Uddrag af \code{\_\_init\_\_.py} for simulationsversionen.]
from guard import Timeout, Skip
from pycsp.greenlets.alternation import choice
from alternation import Alternation
from pycsp.greenlets.channel import ChannelPoisonException, ChannelRetireException
\end{lstlisting}

I \cref{fig:init} kan man se, at funktionerne \code{choice}, \code{ChannelPoisonException} og \code{ChannelRetireException} alle bliver hentet fra \emph{greenlets}"-versionen, mens funktionerne \code{Timeout, Skip} og \code{Alternation} bliver importeret fra samme mappe og derfor er modificerede versioner. For slutbrugeren  vil dette dog ikke være synligt, og han vil blot se \emph{simulation}"-versionen som en selvstændig version på lige fod med de andre tre versioner.

\subsection{Scheduler-klassen}
\label{sec:scheduler}
Med valget af \emph{greenlets}"-versionen som grundversion, og med henblik på at hovedparten af vores ændringer vil være i \sched en, vil vi kort gennemgå dele af klassen \code{Scheduler}.

\begin{lstlisting}[firstnumber=132,stepnumber=5,numbers=left, float, label=fig:scheduling, caption=Uddrag af Scheduler.py i \emph{greenlets}versionen.]
    def getInstance(cls, *args, **kargs):
        '''Static method to have a reference to **THE UNIQUE** instance'''
        if cls.__instance is None:
            # (Some exception may be thrown...)
            # Initialize **the unique** instance
            cls.__instance = object.__new__(cls)

            # Initialize members for scheduler
            cls.__instance.new = []
            cls.__instance.next = []
            cls.__instance.current = None
            cls.__instance.greenlet = greenlet.getcurrent()

            # Timer specific  value = (activation time, process)
            # On update we do a sort based on the activation time
            cls.__instance.timers = []

            # Io specific
            cls.__instance.cond = threading.Condition()
            cls.__instance.blocking = 0
\end{lstlisting}

 I \cref{fig:scheduling} ses et uddrag af initialiseringskoden, der er interessant, fordi det er her alle de interne datastrukturer oprettes. Man kan her se de tre lister af processer, som \sched en har til rådighed til at varetage processkifte. 
 \begin{list}
 \tightlist 
 \item \code{new}: Initieres på linje 140 og består af processer, som lige er blevet planlagt for første gang. Nye processer kan ankomme til listen \code{new} via funktionerne \code{Parallel, Sequence} og \code{Spawn}.
 \item \code{next}: Initieres på linje 141 og indeholder de processer, der er klar til at blive kørt, og som har været kørt på et tidligere tidspunkt. Et eksempel kunne være en proces, der har stoppet sin kørsel for at vente på kommunikation. Processen vil blive placeret i denne liste, når kommunikationen er overstået. 
 \item \code{timers}: Initieres på linje 147 og indeholder de processer, der har tilknyttet en timeout. De skal først planlægges på et senere tidspunkt og venter dermed blot. Hvert element i listen består både af processen samt et tidsstempel for hvornår processen skal genaktiveres. 
 \item \code{blocking}: Initieres på linje 151 og er en variabel. Processer, der venter på IO-operationer, er ikke klar til at blive planlagt, men heller ikke afsluttet. \Sched en kan derfor ikke planlægge dem, men holder styr på antallet af ventende processer vha. denne variabel. Dette bruges bla. for at kunne afgøre om \sched en har planlagt alle processer.
\end{list}

Når \sched en er startet, gennemløber den alle tre lister gentagne gange, indtil de alle er tomme, og der ikke er nogle processer, der er blokeret. Dette betyder at der ikke længere kan komme nye processer til der ønsker at blive lagt på \sched en, og dermed kan den afslutte.

For at markere at vi ikke kun skal  planlægge rækkefølgen
af processerne, men også styre tiden, har vi lavet en
\code{Simulation}"-klasse, der arver fra \code{Scheduler}"-klassen. Alle ændringer
vi skal foretage for at kunne planlægge processer under hensyntagen til tid, vil således være indkapslet i denne klasse. 
Dette har yderligere den fordel, at man tydeligt kan se at alle klasserne i \code{simulation}-versionen arver en \sched~ fra \code{Simulation} og
ikke en \sched \xspace fra \code{greenlets}-versionen.

\subsection{Tid} \label{sec:tid}
For at kunne planlægge begivenheder i \des kræves det at alle processer og \sched en har en global forståelse af tid.  Det er derfor en af hjørnestenene i implementeringen af \des, hvordan tid introduceres til \pycsp.  
Begrebet tid er ikke en del af \csp, men er alligevel blevet introduceret i \pycsp, for  at  kunne tilknytte en timeout til en \code{alternation}. Vi ønsker
at videreudvikle denne struktur til at håndtere tid generelt for alle
processer, og til at fungere med vores tidsmodel, modsat den eksisterende
løsning, hvor \code{time}-modulet benyttes.
 
\CRef{Timeout} viser et eksempel på brugen af tid i det eksisterende \pycsp, hvor en \code{alternation} er villig
til at læse fra kanalenden \code{Cin} i 0,5 sekunder. Hvis ikke der
er modtaget en besked indenfor 0,5 sekunder, accepteres dens \code{timeout-guard},
og processen fortsætter sin kørsel uden at have læst fra kanalen.

\begin{lstlisting}[float=hbtp, 
label=Timeout,caption=Timeout i Alternation (fra dokumentationen til PyCSP)]
Alternation([{Timeout(seconds=0.5):None}, 
             {Cin:None}]).select()
\end{lstlisting}

\inline{Dette skal gennemlæses når deadline/interactive  er skrevet for at se om time modulet skal hives op på et højere niveau}

Fra en brugers synsvinkel repræsenteres tiden internt i \pycsp som realtid, men dette er ikke korrekt. Helt generelt kan computere ikke håndtere kontinuerlige begreber, som realtid er, og \code{time}-modulet, der står for håndtering af tid i Python og dermed \pycsp, kan derfor ikke give brugeren mulighed for at benytte realtid. 

 I stedet tilbyder \code{time}-modulet det, vi vil kalde ``pseudo realtid'', der minder om realtid, men på en række områder afviger fra denne. Den største forskel mellem realtid og pseudo realtid er, at i computere kan tiden ikke være kontinuerlig, men må nødvendigvis være diskritiseret, og som oftest i et fast lille tidsskridt. Vi skal med vores diskrete tidsmodel derfor ikke foretage en konvertering fra kontinuerlig til diskret tid, men i stedet skal foretage en konvertering fra en diskret tidsmodel. hvor tiden stiger med et fast tidsskridt, til en diskret tidsmodel, hvor tiden stiger med variable tidsskridt. Vi skal desuden ændre fremskrivningen af tiden så den  drives af begivenheder og ikke af et eksternt modul i pseudo realtid. \fxnote{Kan vi ryste noget ud med hvorfor det er godt vi ikke skal gå fra kontinuerlig til diskret tid?}
 
 
\subsubsection{Sammenhæng mellem tidsintervaller i realtid og diskret tid}
Da man i \code{time}-modulet har et fast tidsskridt, og
realtiden også er inddelt i faste størrelser
som eks. sekunder, kan man med \code{time}-modulet måle tidsintervaller, der
korresponderer med realtiden. I \des findes der ikke en
sammenhæng til  realtiden, da tiden blot er et tal, der starter som 0, og stiger
i arbitrære tidskridt. Når tiden i \des på denne måde er afkoblet
en relation til realtid, giver det ikke mening at have elementer i simuleringen, der er afhængige af realtid. 
I \pycsp kan man planlægge en timeout til at ske om f.eks. 5 sekunder. I \des findes sekunder som
begreb ikke, men man  angiver i stedet, at når tiden er talt op med 5 tidsskridt, skal
begivenheden ske. Der findes dog ikke en total afkobling mellem tiden i \des og realtiden, for givet et konkret problem, der skal modelleres i \des, vil der altid være en sammenhæng mellem tiderne. Men da denne sammenhæng ikke er fast, skal den defineres eksplicit af brugeren, som f.eks at 5 sekunder i problemet defineres som en stigning i tiden med f.eks 5, 0,5 eller 0,05 i simuleringsmodellen.

Når tiden i \des er uafhængig af realtiden, er der ingen grund til at bruge en kompleks model af tiden, og vi har derfor valgt at repræsentere tiden som et positivt tal, der findes internt i \sched en. Dermed findes der kun en version af tiden, da  \sched en er en singelton. For processer, der ønsker at kende tiden, har vi
introduceret funktionen \code{Now}, der returnerer tiden fra \sched en, når funktionen kaldes. En fordel ved brugen af funktionen \code{Now} som en wrapperfunktion til at bede om tiden i forhold til den eksisterende kode, der direkte kalder \code{time}-modulet, er, at vi frigøres fra en konkret implementering af tid. For fremtiden er det kun funktionen \code{Now}, der skal omskrives, for at hele systemet bruger en anden implementering tid.

I den eksisterende kode har det ikke været tiltænkt, at man ønskede at udskifte implementeringen af tiden, vi skal derfor ændre de steder, hvor \code{time}"-modulet er refereret. Heldigvis bruges \code{time}-modulet kun ved udvælgelse af processer fra \code{timers}-listen (\cref{fig:green:timer}). 
\begin{figure}[hbtp]
\begin{minipage}[c]{\linewidth}
\begin{lstlisting}[firstnumber=204, label=fig:green:timer, caption=Udvælgelse af proces fra listen timers (fra scheduling.py)]
if self.timers and self.timers[0][0] < time.time():
  _,self.current = self.timers.pop(0)
  self.current.greenlet.switch()
\end{lstlisting}
\end{minipage}
\begin{minipage}[c]{\linewidth}
\begin{lstlisting}[firstnumber=124, label=fig:sim:timer, caption=Udvælgelse af proces fra listen timers (fra simulation.py)]
if self.timers and self.timers[0][0] <= Now():
  assert self.timers[0][0] == Now()
  _,self.current = heapq.heappop(self.timers)
  self.current.greenlet.switch()
\end{lstlisting}
\end{minipage}
\end{figure}
Her sammenlignes på linje 204 den første tidsværdi i \code{timers} med det nuværende tidspunkt givet af \code{time}-modulet.
Hvis det nuværende tidspunkt er større end værdien i \code{timers}, udvælges denne proces til at køre næste gang og fjernes fra listen.
For at planlægge begivenheder præcist, skal processerne kunne eksekveres på et specifikt tidspunkt. Dermed har  \sched en i \code{simulations}-versionen behov for at kunne  styre aktiveringen af processerne på et finere niveau end, hvad der er muligt med \code{greenlets}"-versionen.  
I \code{simulation}"-versionen har vi fuld kontrol over tiden, da  den står stille, mens processerne eksekveres, hvorfor dette ikke er et problem.
Vi  tilføjer den begrænsning, at tiden skal være præcist det, der er angivet i \code{timers}, før processen skal aktiveres, og ikke kun større, som angivet i \code{greenlets}-versionen. \CRef{fig:sim:timer} viser udvælgelsen af en proces fra \code{timers} i \code{simulerings}-versionen ved brug af \code{Now}-funktionen, hvor tidspunktet skal være præcist det som processen har angivet.

\subsubsection{Fremskrivning af tid}
I pseudo realtid drives tiden frem af et eksternt modul for at efterligne realtid, der kontinuerligt stiger. I pseudo realtid fremskrives tiden derfor uafhængigt af processernes tilstand og derfor vil et program der med korte mellemrum beder om tiden, få et stigende tidspunkt. I \des  skal tiden i modsætning til realtid stå stille, når processerne er aktive, og kun i forbindelse med en planlagt begivenhed skal tiden drives frem til tidspunktet for denne begivenhed.

\fxnote*{skriv tegning}{Vi kan demonstrere, hvordan den kontinuerlige tid har indflydelse på  \pycsp med et eksempel}. Proces 1 har startet en ny tråd via \code{io-decoratoren} og er derfor blokeret. Proces 2 står i en \code{alternation} med en \code{timeout-guard}. Uafhængigt af den tid, det tager proces 1 at komme ud fra sit blokerede kald, skal proces 2 vide hvornår dens timeout er indtrådt. Dette er implementeret i \code{greenlets}-versionen i \cref{fig:blocking_sleep} på linjerne 242 til 251. Her startes en separat tråd, der vil signalere til \sched en, når tiden for næste begivenhed i \code{timers} listen indtræffer. \Sched en kan nu nøjes med at vente på et signal, som vil komme fra enten en blokeret proces eller den nyoprettede tråd.

Når tiden i \des ikke er drevet af en eksternt modul, er nødvendigheden af en ekstra tråd til  håndtering af tid irrelevant. Først når alle begivenheder til et tidsskridt er eksekveret, skal tiden i \des tælles op. Dette betyder i vores konkrete eksempel, at  så længe proces 1 er blokeret, står tiden stille, og  \sched en venter på dem. Vi kan ikke tælle tiden op, blot fordi nogle processer er blokeret af \code{io-decoratoren}, ligesom vi ikke kan tælle tiden op, blot mens nogle processer eksekvere en anden tilfældig funktion.

Først når alle processer venter i enten \code{timers} listen eller på kommunikation, kan der ikke ske flere begivenheder og tiden sættes op. 
I dette tilfælde  kan \sched en finde tidspunktet for den begivenhed, der ligger tættest på det nuværende tidspunkt,  og springe frem til denne begivenhed. Dette er implementeret i \cref{fig:sim_sleep}.
\begin{figure}[hbtp]
\begin{minipage}[c]{\linewidth}
\begin{lstlisting}[firstnumber=239, label=fig:blocking_sleep, caption=Uddrag af \sched en i \code{Scheduler}]
self.cond.acquire()
if not (self.next or self.new):
    # Waiting on blocking processes or all processes have finished!
    if self.timers:
        # Set timer to lowest activation time
        seconds = self.timers[0][0] - time.time()
        if seconds > 0:
            t = threading.Timer(seconds, self.timer_notify)
            # We don't worry about cancelling, since it makes no 
            # difference if timer_notify is called one more time.
            t.start()
            # Now go to sleep
            self.cond.wait()
    elif self.blocking > 0:
        # Now go to sleep
        self.cond.wait()
    else:
        # Execution finished!
        self.cond.release()
        return
self.cond.release()
\end{lstlisting}
\end{minipage}
\begin{minipage}[c]{\linewidth}
\begin{lstlisting}[firstnumber=158, label=fig:sim_sleep, caption= Uddrag af \sched en i \code{Simulation}]
self.cond.acquire()
if not (self.next or self.new):
  # Waiting on blocking processes
  if self.blocking > 0:
    # Now go to sleep
    self.cond.wait()
  #If there exist only processes in timers we can increment
  elif  not (self.next or self.new or self.blocking): 
      if self.timers:
          # inc timer to lowest activation time
          self._t = self.timers[0][0]
      else:
          # Execution finished!
          self.cond.release()
          return
self.cond.release()  
\end{lstlisting}
\end{minipage}
\end{figure}


\subsection{Funktionen Wait}\label{sec:Wait}
Vi har valgt at anskue planlægningen af en begivenhed til et bestemt tidspunkt, sådan at den proces der skal udføre begivenheden venter indtil tiden for begivnheden er nået, og først her begynde udførslen. Dette vil i praksis være det samme som en planlægning til tidspunktet men det letter implementeringen da vi ikke behøver nogen viden om specifikke begivenheder i vores \sched. 

\begin{lstlisting}[firstnumber=11 , stepnumber=2, numbers=left,float=hbtp, label=fig:simpy:yield, caption= Et yield i \simpy (taget fra Bank05.py i eksemplet fra \simpy)] 
def visit(self,timeInBank): 
  print now(), self.name," Here I am" 
  yield hold,self,timeInBank print now()
  self.name," I must leave" 
\end{lstlisting}
I programmeringssproget \simpy benytter man også denne metode med at lade en proces vente. Dette gøres ved at
foretage kaldet \code{yield}, som sørger for, at processen ikke
fortsætter, før et defineret tidsrum er gået. \CRef{fig:simpy:yield} viser, at en kunde er ankommet til banken, hvorefter kunden printer tiden, foretager et \code{yield}, printer tiden igen og afslutter.  Når processen har kaldt \code{yield}, er tiden steget med værdien af \code{timeInBank}. Grunden til at \simpy kan bruge \code{yield}, der er indbygget i Python og at dette kald afgiver kontrol over processen i et tidsrum, knytter sig til deres implementering af \simpy, hvor de bruger \code{corutine} som underliggende teknologi. Som standard kan  \code{corutine} afgive kontrollen med en proces via \code{yield}, og \simpy behøver derfor kun at sikre, at tiden er talt tilstrækkeligt op, før de returnerer til processen.

 Vi ønsker i \pycsp at have en lignende mulighed for at lade en proces vente. Med \code{greenlet}-modulet af brugertråde kan vi ikke bruge \code{yield}, da denne er specifik for \code{co-rutines}, men funktionaliteten er allerede delvist introduceret via \code{timeout} til \code{alternation} i \emph{greenlets}-versionen af \pycsp. Vi kan derfor bygge videre på denne funktionalitet med funktionen \code{Wait}, der fungerer som timeout, men som kan kaldes af processerne på et vilkårligt tidspunkt.

\begin{lstlisting}[firstnumber=20,float=hbtp, label=fig:wait, caption=Wait i \code{simulering}-versionen.] 
def Wait(seconds):
  Simulation().timer_wait(Simulation().current, seconds)
  t = Now()+seconds
  while Now()<t:
    p = Simulation().getNext() 
    p.greenlet.switch()
\end{lstlisting}

Funktionen \code{Wait} står for at kalde den interne funktion, der er lavet til timeouts, kaldet time\_wait, samt sørge for først at returnere, når tiden er steget til det krævede. \code{Wait} er reelt det eneste værktøj, der skal til for at vente, og dermed planlægge begivenheder ud i fremtiden, og vi har på nuværenede tidspunkt en simpel \des, der kører i realtid. 

\subsection{Timers}  
I \pycsp bruges listen \code{timers} til at placere processer, der venter på en timeout i \code{alternation}. Dette er en niche feature ved \pycsp, som  sjældent bruges, og hvor der sjældent er samlet mange processer på en gang. 
I \cref{sec:Wait} beskriver vi hvordan processer, der ønsker at vente lægges på \code{timers}-listen, og i \cref{sec:tid} beskriver vi, hvordan \sched en kun tæller tiden op, når ingen processer kan foretage sig mere i et givent tidsskridt. \fxwarning{Disse referencer passer ikke!! Jeg er ikke sikker på at vi klart har skrevet det nogen steder}Når tiden tælles op, vil  alle processer enten vente på kommunikation eller befinde sig i listen \code{timers}. De processer, der venter i en \code{alternation} og har tilknyttet en timeout, ligger begge steder. Gennemsnitslængden af listen vil derfor stige voldsomt og i vores version og dermed ændres kravene til hvilken  datastruktur der er bedst egnet. 
Med en kort, sjældent brugt liste vil omkostningerne til oprettelse og vedligeholdelse af en avanceret datastruktur være større end fordelene. Til vores skemaplanlægning  vil en min"-hob være det åbenlyse valg, da  man kan  indsætte elementer i konstant tid og fjerne det mindste element i $O(log\ n)$. Skemaplanlægning er specifikt nævnt i introduktionen til Pythons implementering\footnote{\url{http://docs.python.org/dev/3.0/library/heapq.html}} som eksempel på anvendelsesmulighederne for en hob. 

Da implementeringen af en hob allerede findes i Python i modulet \code{heapq}, som er effektivt implementeret i C, vælger vi at bruge denne. Den eneste handling,
der ikke er implementeret som standard, er fjernelsen af et arbitrært element fra hoben. Dette sker i den eksisterende løsning, når en proces
aktiverer et andet valg i \code{alternation} end timeout. I dette tilfælde skal
processen ikke vente på sin timeout, og derfor skal elementet fjernes fra \code{timers} listen. Her må man, som i
en normal liste, lave en lineær søgning i hoben og derefter genoprette
hob"-egenskaben i listen. Det vil dog ikke tage længere tid, da fjernelse af en timeout i \emph{greenlets}"-versionen på nuværende
tidspunkt bruger en lineær søgning, til at finde elementet der skal
fjernes, og genoprettelsen af hob"-egenskaben også kan gøres i lineær tid.

Det kræver ikke meget omskrivning for at konvertere en liste til en hob, hvilket ses ved at sammenligne \cref{fig:green:timer} linje 205 med \cref{fig:sim:timer} linje 126. 

\subsection{Annekteret kode fra \simpy.}
I vores implementering er der en del overlap med \simpy, da det har været en inspirationskilde til hvordan et simuleringssprog kan udvikles i Python. En del af arbejdet med \simpy har vi kunne bruge direkte i vore implementering efter devisen om ikke at genskrive eksisterende god kode. \Simpy har en \code{Monitor}-klasse, der kan benyttes til dataindsamling, bearbejdning og visualisering. Denne klasse har vi i stor udstrækning genbrugt. Den fungerer ved at gemme en liste af tid/værdi par. Dermed kan man efter endt simulering, analysere  hvordan værdierne har ændret sig over tid. Klassen \code{Monitor} kan bruges direkte af brugere, hvor de selv  står for at at gemme værdier på passende tidspunkter igennem kørslen af programmet. Man ønsker tit at kende længden af en kø, der som oftest er implementeret via en liste. Vi har derfor lavet vores egen liste der kan indeholde en \code{Monitor}. Når længden af listen ændres, gemmes længden af listen i en montor til brug for senere analyse, uden brugeren selv skal stå det. Alternativt kan man lave en separat proces, hvis eneste formål er med en given frekvens at gemme listens længde. Fordelen ved denne løsning er at intervallet er jævnt fordelt, og man derfor lettere kan foretage tidsspecifik statistik.
 
   \fxnote{Hvordan skal events gemmes}
    
  \section{Evaluering}
    \inline{Evaluering af hvordan eksemplet løses efter den valgte 
    implementering benyttes. Inkluderer test+performance}
  \section{Fremtidigt arbejde}
  \section{Opsummering}
