\chapter{CSP og PyCSP}
CSP er et sprog til at beskrive uafhængige processer, der udelukkende udveksler information ved at sende og modtage beskeder over eksplicitte kanaler mellem processerne. Det blev introduceret af \citeauthor{hoare-csp} i \citetitle{hoare-csp}\cite{hoare-csp} og ligger til grund for adskillige praktiske implementationer i et udvalg af programmeringssprog, heriblandt Occam, Java, C++ og Python. I Python hedder udviddelsen PyCSP\cite{pycsp}, og er udviklet i et samarbejde mellem mellem Tromsø og Københavns Universitet, med henblik på at kombinere fordelene fra Python og CSP. 

\section{Kommunikation mellem processer}
Kommunikation mellem to processer i CSP kan kun ske når begge processer er klar til at kommunikere. Hvis den ene proces er klar før den anden er den nødt til at vente på at begge er klar. Herefter kan de kommunikere og så fortsætte. Selve kommunikationen foregår over kanaler. De nævnte implementationer har forskellige typer kanaler, set med henblik hvordan de forbindes til andre processer. Generelt findes der fire 4 typer af kanaler: one-to-one, any-to-one, one-to-any og any-to-any. Forskellen ligger i hvem og hvor mange der kan læse og skrive til en kanal. Det er klart at any-to-any kanalen er den mest generelle og har funktionalitet som de andre. Dette er udnyttet i bl.a. PyCSP hvor det er den eneste kanaltype der er til rådighed. Fælles for dem er at det skal specificeres om en proces skal kunne skrive eller læse fra en kanal\fxnote[inline, nomargin]{fact check på dette}. Hvis to processer skal kunne både læse fra, og skrive til hinanden, skal de have to kanaler for at opnå det. 

en \code{alternation} er en struktur til at foretage valg omkring kommunikation. En \code{alternation} kan indeholde et vilkårligt antal kanalender, samt eventuelt en \code{guard}. Når en \code{alternation} udføres foretages der således et valg mellem de kanaler der er klar eller \code{guarden}. En guard kan enten være en \code{timeout}- eller \code{SKIP-guard}. En \code{timeout-guard} er klar efter en angivent tidspunkt og en \code{SKIP-guard} er altid klar. Dette giver mulighed for konstruktioner som f.eks. "kommuniker hvis der er processer der ønsker at kommunikere med os, ellers fortsæt". 




\subsubsection{noter}
Kanaler, karakteristika for dem\\
Alternations, guards og andre vigtige strukturer\\
Tid i PyCSP\\

3 versioner af PyCSP - greenlets osv.\\
 - indsigt og valg af greenlets her\\

parallel, sequence, spawn\\
buffered channels\\
retire\\





