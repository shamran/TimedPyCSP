\begin{otherlanguage}{english}
\thispagestyle{empty}
  \begin{abstract}
    \fxwarning{Abstract mangler}

Over the last years, multi-core cpu's have become increasingly more common, which has lead to a demand for easy representation of concurrency in application development. This has increased the popularity of CSP, leading to implementations in several common programming languages one of which is Python. 

Most practical representations of time in \pycsp currently breaks with the CSP paradigm, so in this thesis we will explore the possibilities of including a representation of time directly in \pycsp. The goal is to make one or several representations of time that gives a developer the tools needed to handle problems within a timecentric domain. We limit ourselves to uses of time within discrete event simulation, real time planning and interactive planning. For each use we approach the problem using applicable examples to identify key issues and requirements. 

We have found that representations of time in applications are not bound to their problem domain, but rather the time model they apply. As such we have developed two representations of time, one using discrete time, the other using real time. 

%Our representation of discrete time meets all identified demands for application in discrete event simulations. We provide an intuitive and flexible solution, that ... 

%Realtime planning and interactive planning have been joined in one representation as they share time model. The model uses the Earliest Deadline First (EDF) algorithm to prioritize events, with added priority inheritance to handle priority inversion. 

Our solution provides developers with intuitive and flexible representations of time, allowing them to focus more on the actual problem, and less on representation and management of time. 

\end{abstract}
\end{otherlanguage}
%\begin{abstract}
%  bla bla (på dansk)
%\end{abstract}
