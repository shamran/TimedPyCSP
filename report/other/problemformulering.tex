%\chapter{Intro} 
\section*{Problemformulering}
%hvad vi vil lave , hvordan vi vil lave det, hvad det skal resultere i, hvorfor gør vi det. 

Vi ønsker i dette speciale at analysere mulighederne for i praksis at inkludere tre tidsmodeller DES, deadline scheduling og interactive time i pycsp.
%Vi ønsker med eksempler at vise vores udviddelse med praktiske eksempler ledsaget af tegninger og tekst.

Vi ønsker på baggrund af analysen at foretage en hel eller delvis implementation af tidsmodellerne. Med udgangspunkt i eksempler vil vi påvise fordele og ulemper i implementationen, samt foretage en sammenligning med den hidtidige måde at simulere tid på.

Vi ønsker at vores udvidelse skal kunne bruges i et bredere udviklermiljø, og derfor ønsker vi at vores løsning skal være teknisk velfunderet samt have en notation der nemt kan bruges.

%Vi ønsker at udvidde sproget pycsp så det i praksis bliver muligt for processer at kende til tid.
\subsection*{Motivation}
For at modellere de nævnte tidsmodeller i PyCSP idag, er en meget benyttet metode at introducerer barrierer for at definere tidsskridt. Denne løsning kræver en stor omskrivning af et udviklet program, sammenlignet med en version uden en tidsmodel. Dette er ikke ønskværdigt og vi vil gerne udvikle en funktionalitet i PyCSP, så man kan benytte tid intuitivt i et udviklet program.

Der er tidligere blevet arbejdet med at introducere tid i CSP, specifikt i form af Timed CSP. Dette er dog primært et teoretisk arbejde, og har aldrig vundet indpas som en gnængs standtard i praktiske implementationer af CSP. 
Vi tror at en praktisk anvendelig implementation af tidsbegrebet vil have stor betydning for PyCSP's anvendelighed og udbredelse som et værktøj til at løse problemer der naturligt har en dimension af tid. 

\fxnote[inline]{Skal vi komme ind på at man med tid vil kunne resonere om processerne ud fra en stokastisk tidselement.}

\fxnote{Brian kan du komme med mere motivation.}

\subsection*{Resultat}
Projektet skal resultere i en udviddelse af PyCSP der understøtter de nævnte modeller, samt en evaluering af udviddelsen. En vigtig del af resultatet vil være en række forklarende eksempler, der viser ''best practice'' for anvendelsen.
\\
\\
\textbf{Læringsmålene for dette projekt er som følger:}

\begin{itemize}
 \item Identificere problemstillingerne ved at introducere tidsmodeller i PyCSP.
 \item Argumentere for hvilke ændringer  der skal foretages i PyCSP for at introducere tidsmodeller.
 \item Foretage en hel eller delvis implemention i PyCSP af de nævnte ændringer.
 \item Foretage en evaluering de implementerede tidsmodeller sammenholdt med lignende løsninger.
 \item Beskrive de foretagede ændringer så de er trivielle at benytte for folk med kendskab til PyCSP.
 \item Gennem eksempler demonstrere anvendelsen.
\end{itemize}
\end{document}
