\chapter{Introduktion}
Vi vil indledningsvis præsentere konteksten og baggrunden for dette speciale. Herefter vil vi klarlægge hvilke problemer vi ønsker at adressere samt hvordan vores tilgangsvinkel vil være. Afslutningsvis giver vi et overblik over specialets struktur. 

\subsubsection{Baggrund}
Over de sidste par år er multi-kerne cpu'er blevet hyldevarer, hvilket har afledt et stigende behov for at udvikle programmer der kan udnytte flere kerner samtidigt. Dette behov har gjort CSP til et populært sprog, da det gør det let at repræsentere samtidighed samtidig med at det kræver eksplicit udveksling af data, frem for at benytte delte datastrukturer som kræver låsemekanismer eller anden form for kontrol over hvem og der tilgår og hvordan det delte data tilgås. CSP's stigende popularitet har affødt at det er blevet blevet implementeret i flere andre programmeringssprog, og senest har Google lavet sproget Go, der er baseret på CSP. Med den stigende popularitet stiger behovet også for udviddet funktionalitet indenfor niche-områder, heriblandt håndtering af tidsspecifikke problemer. Der er tidligere blevet arbejdet meget med at introducere tid i CSP, men det er primært på det teoretiske plan og har aldrig vundet indpas i de gængse implementationer. 

\subsubsection{Specialets problemformulering og struktur}
Set i lyset af den nuværende mangel på en praktisk anvendelig implementering af tid i CSP, vil vi undersøge om det er muligt at lave en sådan - dvs. en implementering, som kan bruges af udviklere til at løse problemer, der har en naturlig dimension af tid.\fxwarning{Hvad er en "naturlig dimension af tid"?}\fxwarning{Har I defineret "praktisk anvendelig implementering" tilstrækkeligt?}

For at opnå dette vil vi undersøge, hvad der skal til for at introducere følgende tre tidsmodeller i \pycsp: Diskret tid, realtid og interaktiv tid. Disse modeller repræsenterer forskellige anvendelser af tid og dækker tilsammen bredt over tid som helhed. Diskret tid anvendes i stor udstrækning i simuleringer specifikt i discrete event simulation. Realtid benyttes i tidskritiske systemer hvor der er stringente krav om at en given begivenhed er blevet udført inden for en tidsramme. Endeligt bruges interaktiv tid til at specificere at en given begivenhed skal finde sted på et specifikt tidspunkt, eller alternativt inden for en givet tidsrum. 

For hver model vil vi dentificere problemerne ved introduktionen af denne og komme med løsningsforslag som tager udgangspunkt i den praktiske anvendelighed. Disse løsningsforslag vil blive implementeret som en udviddelse af \pycsp. 

Specialet vil derfor være struktureret som følger. I \autoref{chap:csp} vil vi gennemgå CSP og \pycsp med fokus på de dele der er relevante i forhold til at introducere tidsmodeller. I \autorefs{chap:des}, \ref{chap:rtp} og \ref{chap:is} vil vi gennemgå de tre tidsmodeller som beskrevet ovenfor. Desuden vil vi med fokus på den praktiske anvendelighed konstruere eksempler til hver tidsmodel der viser hvordan et tidsspecifikt problem kan løses henholdsvis med og uden vores udviddelse. Vi vil definere eksemplerne så de bedst muligt illustrerer de problemstillinger vi har identificeret ved hver enkelt tidsmodel. Eksemplernes formål er derfor at vise problemstillingerne samt give et klart indblik i hvordan udviddelserne benyttes og hvilke fordele en introduktion af de givne tidsmodeller i \pycsp vil give. Afslutningsvis vil vi foretage en samlet i evaluering og konklusion i \autoref{chap:konklusion} på baggrund af delkonklusionerne i \autorefs{chap:des}, \ref{chap:rtp} og \ref{chap:is}.

%Mål: At lave en praktisk anvendelig udviddelse af pycsp, som kan bruges af udviklere til at løse problemer, der har en naturlig dimension af tid.

%Mål: At undersøge om det er muligt at lave en praktisk anvendelig udviddelse af pycsp, som kan bruges af udviklere til at løse problemer, der har en naturlig dimension af tid.

\section{Vores bidrag}
\section{Termer}

\begin{list}{}{}
\tightlist
\item Scheduler findes ikke som et dansk ord, som kan  dække helt det samme. Vi har valgt at bruge ordet skemaplanlægger.
\item I \des beskriver det enkelte \code{event} en begivenhed, og vi vil i dette speciale bruge ordet begivenhed for en event i \des. 
\item realtid: tid mål i sekunder, minutter osv. TODO: TJEK FOR reel tid.
\item simulering ikke simulation
\item implementering ikke implementation.
\item event: begivenhed
\item Vi har fra koden der vises fjernet kode der ikke er relevant for sammenhængen, som f.eks kald til logging. Linje nr. vil derfor ikke altid passe, men det første linje nr i kodestumpen vil svare til linjenr i source koden.
\item hvad skal emphs og hvad skal med stå med code osv.
\item TODO: vi skal søge og styr på  greenlet vs. greenlets
\item søg igennem grenelts og varianter
\item SKIP-guard, skip-guard \code{skip-guard}? træf et valg mht. både skip og timeout. 
\end{list}


Vi vil bruge tekst markeret med \code{skrivemaskine-font} til at markere variabelnavne, funktioner,  klasser og moduler  der er  der er specifikt for Python.

