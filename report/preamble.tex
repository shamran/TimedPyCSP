\documentclass[a4paper, % papirstørrelse, skal altid med
final,% Når man skal skifte kompileringsmetode mellem draft og final skal man 
      %flytte kommenteringen %  i Draft kommanoerne, nedenfor.
11pt, % standardstørrelse på fonten
openany%, % anvnedes hvis kapitler bare skal starte på den næste side
%oneside,
%article
]{memoir}%
% marginer, memoir to to metoder
% den traditionelle, se memoir manualen
% venstre: 2.5cm, højre: 3.5cm, top: 3cm, bund: 4cm
%her kan margenen sættes manuelt, hvis man har lyst
%\setlrmarginsandblock{3.5cm}{3.5cm}{*}
\setlrmarginsandblock{1cm}{5cm}{*}
%her kan margenen sættes manuelt, hvis man har lyst
\setulmarginsandblock{2cm}{2cm}{*}
% % mere utraditionelt, men hurtigt smartere
% % vi sætter tekstblokken og placerer den
% % værdierne som er angivet svarer ca. til dem anvende i LaTeXbogen
% % 400pt bred og en højde svarende til at forholdet er det gyldne snit
%\settypeblocksize{*}{371pt}{1.618}
% % placer tekstblokken på papriet, kun en værdi skal angives, her sider
% % vi bare hvad forholder mellem marginerne skal være
% \setlrmargins{*}{*}{1.6}
% \setulmargins{*}{*}{1.3}
\checkandfixthelayout % laver forskellige beregninger og sætter de
% almindelige længder op
\usepackage[utf8]{inputenc} % eller utf8 eller ansinew eller ...
\usepackage[english, danish]{babel} %direktiv til at bruge det danske sprog
\selectlanguage{danish}
\usepackage[draft]{fixme} % til at skrive \fixme kommentarer til sig

%\newenvironment{Draft}{\fixme{HUSK at ændre kommenteringen i preamble når man skifter mellem draft og final} }{  }
\let\Draft=\comment \let\endDraft=\endcomment

 
%har vi ord der bliver delt forkert kan de indsættes i hyphernation med bindestreg alle steder hvor ordet kan deles
\hyphenation{tit-len mo-del-len pro-du-ce-rer an-svars-hav-ende om-for-deles for-bind-elses-mulig-heder ska-ber æn-dring-er net-værks-en-hed-er peer backup backup-system fast-track backup-systemer nabo-peers frame-work efter-som Grund-lag-et plads-effek-tivi-teten audits}

\usepackage{nameref}
\usepackage[danish]{varioref} % smarte krydsreferencer via \vref
\usepackage{color}
\definecolor{darkblue}{rgb}{0,0.1,0.5}
\usepackage[colorlinks, linkcolor=darkblue, citecolor=darkblue, urlcolor=darkblue, 
final=true]{hyperref}
%giver mulighed for at hoppe i pdf'filen via referencerne.
%hyperref er ustabil med varieof,
%og skal udkommenteres hvis der opstår problemer.
%specielt skal udkommenteres når man kompilere i final til print

\AtBeginDocument{\let\orgaddcontentsline\addcontentsline
\renewcommand{\addcontentsline}[3]{
        \ifthenelse{\equal{#1}{toc}}
                {\ifthenelse{\equal{#2}{subsection}}
                        {\orgaddcontentsline{#1}{section}{#3}}
                        {\orgaddcontentsline{#1}{#2}{#3}}}
                {\orgaddcontentsline{#1}{#2}{#3}}}}
%
%% Include hyperref for link support
%%\usepackage[dvipdfmx,   % If need dvipdfmx
%%\usepackage[dvips,      % If need dvips
%\usepackage[%           % Fine in most cases
%            pdfpagelabels,hypertexnames=true,
%            plainpages=false,
%            naturalnames=false]{hyperref}
%
%% Configure the hyperlink color
%\usepackage{color}
%\definecolor{darkblue}{rgb}{0,0.1,0.5}
%\hypersetup{colorlinks,
%            linkcolor=darkblue,
%            anchorcolor=darkblue,
%            citecolor=darkblue}
%
%% In the future, it would be nice to use the package
%%
%%    cleveref
%%
%% Right now, it breaks svjour3 and svjour2, so we just have to use
%% autoref and labelformat :(.
%
%% Make sure \autoref puts parentheses around 
%% equations and enumeration items (similar to \eqref)
%% NOTE: This allows us to use \ref instead of \eqref
\labelformat{equation}{\textup{(#1)}}
\labelformat{enumi}{\textup{(#1)}}

%
%% Make sure equations are numbered by section
%\numberwithin{equation}{section}
%
%%% We need to redefine \autoref. We should use 
%%% abbreviations inside the sentence and full names 
%%% at the beginning of sentences. Additionally,
%%% need to handle the plural cases.
%
%% \Autoref is for the beginning of the sentence
\let\orgautoref\autoref
%\providecommand{\Autoref}
%        {\def\equationautorefname{Equation}%
%         \def\figureautorefname{Figure}%
%         \def\subfigureautorefname{Figure}%
%         \def\sectionautorefname{Section}%
%         \def\subsectionautorefname{Section}%
%         \def\subsubsectionautorefname{Section}%
%         \def\Itemautorefname{Item}%
%         \def\tableautorefname{Table}%
%         \orgautoref}
%
%% \Autorefs is plural for the beginning of the sentence
%\providecommand{\Autorefs}
%        {\def\equationautorefname{Equations}%
%         \def\figureautorefname{Figures}%
%         \def\subfigureautorefname{Figures}%
%         \def\sectionautorefname{Sections}%
%         \def\subsectionautorefname{Sections}%
%         \def\subsubsectionautorefname{Sections}%
%         \def\Itemautorefname{Items}%
%         \def\tableautorefname{Tables}%
%         \orgautoref}
%
%% \autoref is used inside a sentence 
%% (this is a renew of the standard)
\renewcommand{\autoref}
        {\def\equationautorefname{Ligning}%
         \def\figureautorefname{Figur}%
         \def\subfigureautorefname{Figur}%
         \def\sectionautorefname{Afsnit}%
         \def\subsectionautorefname{Afsnit}%
         \def\subsubsectionautorefname{Afsnit}%
         \def\Itemautorefname{item}%
         \def\tableautorefname{Tabel}%
         \orgautoref}
%
%% \autorefs is plural for inside a sentence
%\providecommand{\autorefs}
%        {\def\equationautorefname{Eqs.}%
%         \def\figureautorefname{Figs.}%
%         \def\subfigureautorefname{Figs.}%
%         \def\sectionautorefname{Sects.}%
%         \def\subsectionautorefname{Sects.}%
%         \def\subsubsectionautorefname{Sects.}%
%         \def\Itemautorefname{items}%
%         \def\tableautorefname{Tables}%
%         \orgautoref}



\usepackage[style=numeric-comp,sorting=nyt]{biblatex}
\DefineBibliographyStrings{danish}{%
references      = {Litteraturliste},
bibliography    = {Litteraturliste},
urlseen         = {Lokaliseret d\adddot},
andothers 		= {m\adddot fl\adddot},
typeeditor      = {{udgiver}{udg\adddot}},
typeeditors 	= {{udgivere}{udg\adddot}},
}
\usepackage[babel, danish=quotes]{csquotes} %bedrer mulighed for at anførselstegn

\usepackage[T1]{fontenc} % bedre orddeling og ofte påkrævet at
% forskellige fonte
%\usepackage{fourier} % eller mathpazo eller lignende, eller fjern den
% for at få standard fonten
% sætter nogle default værdier vedr. floats

\let\newfloat\relax % memoir har allerede defineret denne men det gør
% float pakken også
\usepackage{float}
\restylefloat{table}
\floatevery{table}{\centering\small} % alle tabel floats centreres og
% skrives i \small
\restylefloat{figure}
\floatevery{figure}{\centering} % automatisk centrering af alle
% figurer
% float environments får ’htbp’ som standard placerings værdier når
% man ikke har angivet noget

\makeatletter
\renewcommand\fps@figure{hbtp}
\renewcommand\fps@table{hbtp}
\makeatother
\usepackage{amsmath,amssymb} % bedre matematik og ekstra fonte
\usepackage{textcomp} % adgagn til tekstsymboler
\usepackage[notcite,notref]{showkeys} % viser labels i margin,
% udkommenteres for at fjerne, eller
% anvend option final


\setsecnumdepth{section} % eller hvor dybt man nu ønsker at har
% overskrifterne nummereret
\maxsecnumdepth{section}
\settocdepth{subsection} % hvor dybt ned vi ønsker ting med i

% OPSÆTNING AF SÆTNINGER etc.
\usepackage[amsmath,thmmarks]{ntheorem} % bedre fleksimibitet
\usepackage[final]{graphicx} % pakke til inklusion af grafik
\usepackage{epsfig}

\makeindex % hvis man ønsker at lave et index
%\usepackage{pstricks}
\usepackage{tikz}

\usepackage{xspace}
%
\newcommand{\cref}[1]{\autoref{#1}\xspace\pageref{#1}} %ændres fra pageref til vpageref in final
\newcommand{\tcite}[1]{\citetitle{#1}\xspace\cite{#1}}

\newcommand{\des}[0]{discrete event simulation\xspace}
\newcommand{\Des}[0]{Discrete event simulation\xspace}
\newcommand{\ds}[0]{Deadline scheduling\xspace}
\newcommand{\is}[0]{Interaktiv scheduling\xspace}
\newcommand{\inline}[1]{\fxnote[inline]{#1}}
\newcommand{\simpy}[0]{SimPy\xspace}
\newcommand{\pycsp}[0]{PyCSP\xspace}
\newcommand{\csp}[0]{CSP\xspace}
% OPSÆTNING TIL INKLUDERING AF SOURCE CODE
\usepackage[final]{listings}
\renewcommand{\lstlistingname}{kodestump}
\renewcommand{\lstlistlistingname}{Kodestumper}
\usepackage{courier}
 \lstset{ 
         basicstyle=\footnotesize\ttfamily, % Standardschrift
         %numbers=left,               % Ort der Zeilennummern
         numberstyle=\tiny,          % Stil der Zeilennummern
         %stepnumber=2,               % Abstand zwischen den Zeilennummern
         numbersep=5pt,              % Abstand der Nummern zum Text
         tabsize=2,                  % Groesse von Tabs
         extendedchars=true,         %
         breaklines=false,            % Zeilen werden Umgebrochen
         keywordstyle=\color{red},
                frame=b,         
         keywordstyle=[1]\textbf,    % Stil der Keywords
         keywordstyle=[2]\textbf,    %
         keywordstyle=[3]\textbf,    %
         keywordstyle=[4]\textbf,   %\sqrt{\sqrt{}} %
         stringstyle=\color{white}\ttfamily, % Farbe der String
         showspaces=false,           % Leerzeichen anzeigen ?
         showtabs=false,             % Tabs anzeigen ?
         xleftmargin=17pt,
         framexleftmargin=17pt,
         framexrightmargin=5pt,
         framexbottommargin=4pt,
         %backgroundcolor=\color{lightgray},
         showstringspaces=false      % Leerzeichen in Strings anzeigen ?        
 }
 \lstloadlanguages{% Check Dokumentation for further languages ...
         [Visual]Basic,
         Pascal,
         C,
         C++,
         XML,
         HTML,
         PYTHON,
 }
\lstset{language=Python}
\lstset{emph={@process, @choise, Alternation, Skip,Timeout,Parallel,Sequence, Spawn,@io,ChannelPoisonException, ChannelRetireException},emphstyle=\underbar}


    %\DeclareCaptionFont{blue}{\color{blue}} 

  %\captionsetup[lstlisting]{singlelinecheck=false, labelfont={blue}, textfont={blue}}
  \usepackage{caption}
\DeclareCaptionFont{white}{\color{white}}
\DeclareCaptionFormat{listing}{\colorbox[cmyk]{0.43, 0.35, 0.35,0.01}{\parbox{\textwidth}{\hspace{15pt}#1#2#3}}}
\captionsetup[lstlisting]{format=listing,labelfont=white,textfont=white, singlelinecheck=false, margin=0pt, font={bf,footnotesize}}




\usepackage{pdfpages}

%\includeonly{../litteratur}




