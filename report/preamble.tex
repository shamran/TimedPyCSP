
\documentclass[a4paper, % papirstørrelse, skal altid med
final,% Når man skal skifte kompileringsmetode mellem draft og final skal man 
      %flytte kommenteringen %  i Draft kommanoerne, nedenfor.
11pt, % standardstørrelse på fonten
%openany,%, % anvnedes hvis kapitler bare skal starte på den næste side
oneside%,
%article
]{memoir}%
% marginer, memoir to to metoder
% den traditionelle, se memoir manualen
% venstre: 2.5cm, højre: 3.5cm, top: 3cm, bund: 4cm
%her kan margenen sættes manuelt, hvis man har lyst
%\setlrmarginsandblock{3.5cm}{3.5cm}{*}


\setlrmarginsandblock{1.5cm}{3.5cm}{*}
\setulmarginsandblock{3cm}{4cm}{*}

%her kan margenen sættes manuelt, hvis man har lyst


% % mere utraditionelt, men hurtigt smartere
% % vi sætter tekstblokken og placerer den
% % værdierne som er angivet svarer ca. til dem anvende i LaTeXbogen
% % 400pt bred og en højde svarende til at forholdet er det gyldne snit
%%\settypeblocksize{*}{371pt}{1.618}
% % placer tekstblokken på papriet, kun en værdi skal angives, her sider
% % vi bare hvad forholder mellem marginerne skal være
% \setlrmargins{*}{*}{1.6}
% \setulmargins{*}{*}{1.3}
\checkandfixthelayout % laver forskellige beregninger og sætter de
% almindelige længder op

%-------definitions-----
\newcommand{\Author}{Simon Christiano Bognolo og Rasmus Ebdrup Sørensen} 
\newcommand{\Title}{Timed PyCSP}
\newcommand{\Subject}{Introducing discrete event simulation and real time scheduling in PyCSP}
\newcommand{\Keywords}{Priority, Deadlines, Real-time, Discrete events, Scheduling}
%-----------------------


\title{\Title}
\author{\Author}


\setsecnumdepth{subsection} % Hvor dybt man nu ønsker at vi har
\maxsecnumdepth{subsection}% overskrifterne nummereret
\settocdepth{subsection} % hvor dybt ned vi ønsker ting med i
 
\usepackage[T1]{fontenc} % bedre orddeling og ofte påkrævet at
\usepackage[utf8]{inputenc} % eller utf8 eller ansinew eller ...
\usepackage[english, danish]{babel} %direktiv til at bruge det danske sprog

\renewcommand\danishhyphenmins{22} % bedre orddeling
\addto\captionsdanish{
\renewcommand\contentsname{Indholdsfortegnelse}
\renewcommand\appendixname{Appendix}
}
\selectlanguage{danish}

%har vi ord der bliver delt forkert kan de indsættes i hyphernation med bindestreg alle steder hvor ordet kan deles
\hyphenation{trans-port-bånd-et tit-len mo-del-len pro-du-ce-rer an-svars-hav-ende om-for-deles for-bind-elses-mulig-heder ska-ber æn-dring-er net-værks-en-hed-er peer backup backup-system fast-track backup-systemer nabo-peers frame-work efter-som Grund-lag-et plads-effek-tivi-teten audits soft-ware dead-line-Excep-tion}



\usepackage{amsmath,amssymb} % bedre matematik og ekstra fonte
\usepackage{textcomp} % adgagn til tekstsymboler
\usepackage[notcite,notref]{showkeys} % viser labels i margin,
% udkommenteres for at fjerne, eller
% anvend option final

% OPSÆTNING AF SÆTNINGER etc.
\usepackage[amsmath,thmmarks]{ntheorem} % bedre fleksimibitet
\usepackage[final]{graphicx} % pakke til inklusion af grafik
\usepackage{epsfig}

\makeindex % hvis man ønsker at lave et index
%\usepackage{pstricks}
\usepackage{tikz}
\usepackage{xspace} %Allow for a space after a new command in the right place
\usepackage{longtable} %allow for tables that span more than one page 
\usepackage[section]{placeins}%pakke til at stoppe float fra at drifte for længere end til \FloatBarrier. Med en section i option ændrers alle sections til at sætte en barriere.
 
\makeatletter
    \declare@shorthand{danish}{"/}{\nobreak\discretionary{/}{}{/}\bbl@allowhyphens}%Tillader orddeling på tværs af / ved at skrive "/
    \declare@shorthand{danish}{"_}{\nobreak\discretionary{\_}{}{\_}\bbl@allowhyphens}%Tillader orddeling på tværs af _ ved at skrive "_
\makeatother

\usepackage{nameref}
\usepackage[danish]{varioref} % smarte krydsreferencer via \vref
\usepackage{color}
\definecolor{darkblue}{rgb}{0,0.1,0.5}
\usepackage{hycolor}
\usepackage[draft, danish,layout=margin,innerlayout=inline]{fixme} % til at skrive \fixme kommentarer til sig

\fxfatal[layout=index,target=settoc]{Hvor meget skal med i toc, og hvad skal stå med tal. preamble linje 47.}


\usepackage[colorlinks%
           ,linkcolor=darkblue%
           ,citecolor=darkblue%
           ,urlcolor=darkblue%
           ,pdfusetitle%
           ,pdfstartview=FitBH%fit width of page to the window
           ,plainpages=false%
           ]{hyperref}%

\hypersetup{%
  pdfauthor={\Author},%
  pdftitle={\Title},%
  pdfsubject={\Subject},%  
  pdfkeywords={\Keywords},%  
}%
%giver mulighed for at hoppe i pdf'filen via referencerne.
%hyperref er ustabil med varieof,
%og skal udkommenteres hvis der opstår problemer.
%specielt skal udkommenteres når man kompilere i final til print

%% Make sure \autoref puts parentheses around 
%% equations and enumeration items (similar to \eqref)
%% NOTE: This allows us to use \ref instead of \eqref
\labelformat{equation}{\textup{(#1)}}
\labelformat{enumi}{\textup{(#1)}}
%\usepackage[draft,author=Dorte]{pdfcomment}


\newcommand{\inline}[1]{\fxwarning[inline,nomargin]{#1}}
\newcommand{\kv}[1]{\fxnote*{Kendt viden?}{#1}}
\newcommand{\ikkef}[1]{\fxnote*{Ikke forstået, gør dataloger?}{#1}}

%\def\figurename{Figur}%
\def\lstlistingname{Kodestump}%


%%% We need to redefine \autoref. We should use 
%%% abbreviations inside the sentence and full names 
%%% at the beginning of sentences. Additionally,
%%% need to handle the plural cases.
%
%% \Autoref is for the beginning of the sentence
\let\orgautoref\autoref
\providecommand{\Autoref}
        {\def\chapterautorefname{Kapitel}% 
         \def\equationautorefname{Ligning}%
         \def\figureautorefname{Figur}%
         \def\figurename{Figur}%
         \def\subfigureautorefname{Figur}%
         \def\sectionautorefname{Afsnit}%
         \def\subsectionautorefname{Afsnit}%
         \def\subsubsectionautorefname{Afsnit}%
         \def\Itemautorefname{Item}%
         \def\tableautorefname{Tabel}%
         \def\lstlistingname{Kodestump}%
         \orgautoref}
%
%% \Autorefs is plural for the beginning of the sentence
%
\providecommand{\Autorefs}
        {\def\equationautorefname{Ligningerne}%
         \def\figureautorefname{Figurerne}%
         \def\subfigureautorefname{}%
         \def\sectionautorefname{Afsnittene}%
         \def\subsectionautorefname{Afsnittene}%
         \def\subsubsectionautorefname{Afsnittene}%
         \def\Itemautorefname{Items}%
         \def\tableautorefname{Tabellerne}%
         \def\chapterautorefname{Kapitlerne}% 
         \orgautoref}


%% \autoref is used inside a sentence 
%% (this is a renew of the standard)
\renewcommand{\autoref}
        {\def\chapterautorefname{kapitel}% 
         \def\equationautorefname{ligning}%
         \def\figureautorefname{figur}%
         \def\figurename{Figur}%
         \def\subfigureautorefname{figur}%
         \def\sectionautorefname{afsnit}%
         \def\subsectionautorefname{afsnit}%
         \def\subsubsectionautorefname{afsnit}%
         \def\Itemautorefname{item}%
         \def\tableautorefname{tabel}%
         \def\lstlistingname{kodeuddrag}%
         \orgautoref}
%figurerne
%% \autorefs is plural for inside a sentence
\providecommand{\autorefs}
        {\def\equationautorefname{ligningerne}%
         \def\figureautorefname{figurerne}%
         \def\subfigureautorefname{}%
         \def\sectionautorefname{afsnittene}%
         \def\subsectionautorefname{afsnittene}%
         \def\subsubsectionautorefname{afsnittene}%
         \def\Itemautorefname{items}%
         \def\tableautorefname{tabellerne}%
         \def\chapterautorefname{kapitlerne}% 
         \orgautoref}



\usepackage[style=numeric-comp,sorting=nyt]{biblatex}
\defbibheading{bibliography}{%
    \chapter{Litteraturliste}}
\DefineBibliographyStrings{danish}{%
references      = {Litteraturliste},
bibliography    = {Litteraturliste},
urlseen         = {Lokaliseret d\adddot},
andothers 		= {m\adddot fl\adddot},
typeeditor      = {{udgiver}{udg\adddot}},
typeeditors 	= {{udgivere}{udg\adddot}},
}



 
\usepackage[babel, danish=quotes]{csquotes} %bedrer mulighed for at anførselstegn


\let\newfloat\relax % memoir har allerede defineret denne men det gør float pakken også
\usepackage{float}
\restylefloat{table}
\floatevery{table}{\centering\small} % alle tabel floats centreres og
% skrives i \small
\restylefloat{figure}
\floatevery{figure}{\centering} % automatisk centrering af alle
% figurer
% float environments får ’htbp’ som standard placerings værdier når
% man ikke har angivet noget

\makeatletter
\renewcommand\fps@figure{hbtp}
\renewcommand\fps@table{hbtp}
\makeatother



\fxfatal[layout=index,target=cref]{Ændre pageref til vpageref}
\newcommand{\cref}[1]{\autoref{#1}\textcolor{darkblue}{~på side}~\pageref{#1}} %ændres fra pageref til vpageref in final  \vpageref{#1}}
\newcommand{\CRef}[1]{\Autoref{#1}\textcolor{darkblue}{~på side}~\pageref{#1}} %ændres fra pageref til vpageref in final

\newcommand{\nref}[1]{\nameref{#1}\textcolor{darkblue}{~på side}~\pageref{#1}}
\newcommand{\tcite}[1]{\citetitle{#1}\xspace\cite{#1}}

\newcommand{\des}[0]{DES\xspace}
\newcommand{\pdes}[0]{PDES\xspace}
\newcommand{\Des}[0]{DES\xspace}
\newcommand{\ds}[0]{RTP\xspace}
\newcommand{\is}[0]{IP\xspace}

\newcommand{\simpy}[0]{SimPy\xspace}
\newcommand{\pycsp}[0]{PyCSP\xspace}
\newcommand{\csp}[0]{CSP\xspace}

\newcommand{\sched}[0]{skemaplanlægger}
\newcommand{\Sched}[0]{Skemaplanlægger}


\newcommand{\code}[1]{\texttt{#1}\xspace}

%\definecolor{darkred}{rgb}{.647,.129,.149}
\definecolor{shadecolor}{rgb}{1,0.8,1}%{gray}{.87}


% usage: 
% \mycite[page no]{books bibname}{cited text}
% or
% \mycite{source}{cited text}
\def\myempty{}
\newcommand*{\mycite}[3][]{\begin{quote}\textit{``#3''~}%
    \def\temp{#1}%
    \ifx\temp\myempty \citep{#2}%
    \else \cite[#1]{#2}%
    \fi  \end{quote}}

%Overskrifter til tabeller - bliver centreret
\newcommand\mc[1]{\multicolumn{1}{c}{{\textbf #1}}} % sparer plads
\newcommand\nc[1]{\multicolumn{1}{c}{#1}} % sparer plads
% OPSÆTNING TIL INKLUDERING AF SOURCE CODE
%\usepackage[final]{listings}
\usepackage{listings}
\renewcommand{\lstlistlistingname}{Kodestumper}
\usepackage{courier}
 \lstset{ 
         basicstyle=\footnotesize\ttfamily, % Standardschrift
         numbers=left,               % Ort der Zeilennummern
         numberstyle=\tiny,          % Stil der Zeilennummern
         stepnumber=2,               % Abstand zwischen den Zeilennummern
         numbersep=5pt,              % Abstand der Nummern zum Text
         tabsize=2,                  % Groesse von Tabs
         extendedchars=true,         %
         breaklines=false,            % Zeilen werden Umgebrochen
         keywordstyle=\color{red},
                frame=b,         
         keywordstyle=[1]\textbf,    % Stil der Keywords
         keywordstyle=[2]\textbf,    %
         keywordstyle=[3]\textbf,    %
         keywordstyle=[4]\textbf,   %\sqrt{\sqrt{}} %
         stringstyle=\color{darkgray}\ttfamily, % Farbe der String
         showspaces=false,           % Leerzeichen anzeigen ?
         showtabs=false,             % Tabs anzeigen ?
         xleftmargin=17pt,
         framexleftmargin=17pt,
         framexrightmargin=5pt,
         framexbottommargin=4pt,
         %backgroundcolor=\color{lightgray},
         showstringspaces=false      % Leerzeichen in Strings anzeigen ?        
 }
 \lstloadlanguages{% Check Dokumentation for further languages ...
         [Visual]Basic,
         Pascal,
         C,
         C++,
         XML,
         HTML,
         PYTHON,
 }
\lstset{language=Python}
\lstset{emph={@process, @choise,choice, Alternation, Skip,Timeout,Parallel,Sequence, Spawn,@io,ChannelPoisonException, ChannelRetireException},emphstyle=\underbar}


    %\DeclareCaptionFont{blue}{\color{blue}} 

  %\captionsetup[lstlisting]{singlelinecheck=false, labelfont={blue}, textfont={blue}}
\usepackage{caption}
\DeclareCaptionFont{white}{\color{white}}
\DeclareCaptionFormat{listing}{\colorbox[cmyk]{0.43, 0.35, 0.35,0.01}{\parbox{\textwidth}{\hspace{15pt}#1#2#3}}}
\captionsetup[lstlisting]{format=listing,labelfont=white,textfont=white, singlelinecheck=false, margin=0pt, font={bf,footnotesize}}


\let\oldlstlisting=\lstlisting
\renewcommand{\lstlisting}{\SingleSpacing\oldlstlisting}
 


\usepackage{pdfpages}
\usepackage{cellspace}
\addtolength\cellspacetoplimit{4pt}
\addtolength\cellspacebottomlimit{4pt}


%\includeonly{../litteratur}




