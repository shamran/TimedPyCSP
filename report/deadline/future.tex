\section{Fremtidigt arbejde}\label{sec:deadlineFuture}
Vi har i dette kapitel opstillet en basal model for en RTP-implementering i \pycsp. Implementeringen er foretaget med henblik på blot at vise muligheden for en sådan model, og der er flere udviddelser som vi mener vil kunne forbedre modellen såfremt de kan implementeres. Vi vil i dette afsnit gennemgå nogle forbedringer som vi mener er interessante, men som ikke er inddraget i den basale implementering. 

\subsubsection{Evaluering af effektivitet}
Vi har med eksempler og test vist, at den implementerede løsning fungerer teoretisk korrekt. Vi har ingen test af om vi reelt når flere deadlines med vores RTP-løsning end vi ville uden. Vi har ikke nogle reelle tal for hvor meget tid vi bruger på at evaluere hvilken proces der skal aktiveres, samt opretholde de metadata der skal til for at foretage denne vurdering. En grundig analyse af tidsforbruget i de administrative dele af vores implementering ville derfor være interessant at udføre, så man bl.a. kan udlede generelle retningslinier for hvor ofte det vurderes hvilken proces der skal køre og hvor beregningstung hver proces bør være for at opnå den bedste ydelse. 

\subsubsection{Estimater for udførselstid}
Den primære begrænsning i vores løsning er manglen på at kunne evaluere hvor lang tid en proces eller dele af en proces tager at udføre. Hvis vi kunne udvikle en løsning der kunne foretage estimater af processers udførselstid ville vi kunne vælge en anden udvælgelsesalgoritme, som f.eks LL frem for EDF. Derved ville udvælgelsen af hvilken proces der skal aktiveres blive mere præcis. Ligeledes vil muligheden for at vurdere hvor langt en proces er nået i sin udførsel også være meget brugbart i forbindelse med prioritetsnedarvning. I vores implementering nedarver vi prioritet til alle processer som har mulighed for at opfylde en afhængighed. Såfremt vi kan vurdere hvilken proces der er tættest på at kunne opfylde afhængigheden, kan vi nøjes med at nedarve prioriteten til denne proces. Dette vil afhjælpe problemet med prioritetsdevaluering som nævnt i \cref{sec:aendring-af-prioritet}.

\subsubsection{Håndtering af forskellige typer deadlines}
I den nuværende løsning håndterer vi alle deadlines ens. Det er der fordele og ulemper ved, hvor en fordel er, at udvikleren får maksimal kontrol over hvad der skal ske såfremt en given deadline overskrides. Vi håndterer overskridelser af deadlines ved at kaste en exception når det sker. Dette er måske ikke altid ønskværdigt hvis det er en soft deadline der overskrides, da processen derved afbrydes. Det kunne tænkes at der er tilfælde hvor det er mere hensigtsmæssigt at udføre processen helt, og først her give besked om at den ikke nåede sin deadline.

\fxerror*{Skal dette med}{I en fremtidig version ville man kunne udvide muligheden med en hybridversion, der skulle kunne håndtere processer med soft deadlines, der skal markeres, og kalde exceptions ved processer med hard deadline. Processen kunne f.eks have tilknyttet dens type af deadline. Systemet kan så reagere passende efter typen af deadlines, så soft deadlines blev færdigbehandlet, mens hard deadlines resulterede i en exception.}


\subsubsection{Udviklerbestemte prioriteter}
På nuværende tidspunkt har alle processerne den samme prioritet inden de planlægges, og deres prioritet afhænger udelukkende af deres deadline. Det kunne være interessant at undersøge om man kan bruge en anden \sched, der kan håndtere at processerne har forskellige prioriteter inden de blev planlagt.

Hvis muligheden  for at differentiere processerne blev implementeret, kunne det være spændene hvis man kunne udvide \sched, så udvikleren kunne angive et kritisk sæt af processer, for hvilke det kunne garanteres at de ikke ville overskrider deres deadline. 

%\subsubsection{Implementering i andre \pycsp versioner}
%Det kunne være interessant at undersøge mulighederne for for at lave en implementering af RTP i andre versioner af \pycsp. Dette vil åbne op for de muligheder der er tilknyttet de forskellige implementeringer, hvilket vil gøre RTP mere praktisk anvendelig. Specielt er muligheden for brug af flere processorer med enten threads\footnote{Ved brug af eksterne moduler}- eller process-versionen. Dette vil dog, som tidligere nævnt, kræve at skemaplanlægningen flyttes til operativsystems skemaplanlægger og bliver derved 


%Test af effektivitet

%Estimater for kørselstid for processer

%Evaluering af hvor langt en proces er fra at være færdig

%Implemementering i process-version
\fxnote{Udvide kanaler til at være alternations med timeouts}
